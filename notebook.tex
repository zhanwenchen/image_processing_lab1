
% Default to the notebook output style

    


% Inherit from the specified cell style.




    
\documentclass[11pt]{article}

    
    
    \usepackage[T1]{fontenc}
    % Nicer default font (+ math font) than Computer Modern for most use cases
    \usepackage{mathpazo}

    % Basic figure setup, for now with no caption control since it's done
    % automatically by Pandoc (which extracts ![](path) syntax from Markdown).
    \usepackage{graphicx}
    % We will generate all images so they have a width \maxwidth. This means
    % that they will get their normal width if they fit onto the page, but
    % are scaled down if they would overflow the margins.
    \makeatletter
    \def\maxwidth{\ifdim\Gin@nat@width>\linewidth\linewidth
    \else\Gin@nat@width\fi}
    \makeatother
    \let\Oldincludegraphics\includegraphics
    % Set max figure width to be 80% of text width, for now hardcoded.
    \renewcommand{\includegraphics}[1]{\Oldincludegraphics[width=.8\maxwidth]{#1}}
    % Ensure that by default, figures have no caption (until we provide a
    % proper Figure object with a Caption API and a way to capture that
    % in the conversion process - todo).
    \usepackage{caption}
    \DeclareCaptionLabelFormat{nolabel}{}
    \captionsetup{labelformat=nolabel}

    \usepackage{adjustbox} % Used to constrain images to a maximum size 
    \usepackage{xcolor} % Allow colors to be defined
    \usepackage{enumerate} % Needed for markdown enumerations to work
    \usepackage{geometry} % Used to adjust the document margins
    \usepackage{amsmath} % Equations
    \usepackage{amssymb} % Equations
    \usepackage{textcomp} % defines textquotesingle
    % Hack from http://tex.stackexchange.com/a/47451/13684:
    \AtBeginDocument{%
        \def\PYZsq{\textquotesingle}% Upright quotes in Pygmentized code
    }
    \usepackage{upquote} % Upright quotes for verbatim code
    \usepackage{eurosym} % defines \euro
    \usepackage[mathletters]{ucs} % Extended unicode (utf-8) support
    \usepackage[utf8x]{inputenc} % Allow utf-8 characters in the tex document
    \usepackage{fancyvrb} % verbatim replacement that allows latex
    \usepackage{grffile} % extends the file name processing of package graphics 
                         % to support a larger range 
    % The hyperref package gives us a pdf with properly built
    % internal navigation ('pdf bookmarks' for the table of contents,
    % internal cross-reference links, web links for URLs, etc.)
    \usepackage{hyperref}
    \usepackage{longtable} % longtable support required by pandoc >1.10
    \usepackage{booktabs}  % table support for pandoc > 1.12.2
    \usepackage[inline]{enumitem} % IRkernel/repr support (it uses the enumerate* environment)
    \usepackage[normalem]{ulem} % ulem is needed to support strikethroughs (\sout)
                                % normalem makes italics be italics, not underlines
    

    
    
    % Colors for the hyperref package
    \definecolor{urlcolor}{rgb}{0,.145,.698}
    \definecolor{linkcolor}{rgb}{.71,0.21,0.01}
    \definecolor{citecolor}{rgb}{.12,.54,.11}

    % ANSI colors
    \definecolor{ansi-black}{HTML}{3E424D}
    \definecolor{ansi-black-intense}{HTML}{282C36}
    \definecolor{ansi-red}{HTML}{E75C58}
    \definecolor{ansi-red-intense}{HTML}{B22B31}
    \definecolor{ansi-green}{HTML}{00A250}
    \definecolor{ansi-green-intense}{HTML}{007427}
    \definecolor{ansi-yellow}{HTML}{DDB62B}
    \definecolor{ansi-yellow-intense}{HTML}{B27D12}
    \definecolor{ansi-blue}{HTML}{208FFB}
    \definecolor{ansi-blue-intense}{HTML}{0065CA}
    \definecolor{ansi-magenta}{HTML}{D160C4}
    \definecolor{ansi-magenta-intense}{HTML}{A03196}
    \definecolor{ansi-cyan}{HTML}{60C6C8}
    \definecolor{ansi-cyan-intense}{HTML}{258F8F}
    \definecolor{ansi-white}{HTML}{C5C1B4}
    \definecolor{ansi-white-intense}{HTML}{A1A6B2}

    % commands and environments needed by pandoc snippets
    % extracted from the output of `pandoc -s`
    \providecommand{\tightlist}{%
      \setlength{\itemsep}{0pt}\setlength{\parskip}{0pt}}
    \DefineVerbatimEnvironment{Highlighting}{Verbatim}{commandchars=\\\{\}}
    % Add ',fontsize=\small' for more characters per line
    \newenvironment{Shaded}{}{}
    \newcommand{\KeywordTok}[1]{\textcolor[rgb]{0.00,0.44,0.13}{\textbf{{#1}}}}
    \newcommand{\DataTypeTok}[1]{\textcolor[rgb]{0.56,0.13,0.00}{{#1}}}
    \newcommand{\DecValTok}[1]{\textcolor[rgb]{0.25,0.63,0.44}{{#1}}}
    \newcommand{\BaseNTok}[1]{\textcolor[rgb]{0.25,0.63,0.44}{{#1}}}
    \newcommand{\FloatTok}[1]{\textcolor[rgb]{0.25,0.63,0.44}{{#1}}}
    \newcommand{\CharTok}[1]{\textcolor[rgb]{0.25,0.44,0.63}{{#1}}}
    \newcommand{\StringTok}[1]{\textcolor[rgb]{0.25,0.44,0.63}{{#1}}}
    \newcommand{\CommentTok}[1]{\textcolor[rgb]{0.38,0.63,0.69}{\textit{{#1}}}}
    \newcommand{\OtherTok}[1]{\textcolor[rgb]{0.00,0.44,0.13}{{#1}}}
    \newcommand{\AlertTok}[1]{\textcolor[rgb]{1.00,0.00,0.00}{\textbf{{#1}}}}
    \newcommand{\FunctionTok}[1]{\textcolor[rgb]{0.02,0.16,0.49}{{#1}}}
    \newcommand{\RegionMarkerTok}[1]{{#1}}
    \newcommand{\ErrorTok}[1]{\textcolor[rgb]{1.00,0.00,0.00}{\textbf{{#1}}}}
    \newcommand{\NormalTok}[1]{{#1}}
    
    % Additional commands for more recent versions of Pandoc
    \newcommand{\ConstantTok}[1]{\textcolor[rgb]{0.53,0.00,0.00}{{#1}}}
    \newcommand{\SpecialCharTok}[1]{\textcolor[rgb]{0.25,0.44,0.63}{{#1}}}
    \newcommand{\VerbatimStringTok}[1]{\textcolor[rgb]{0.25,0.44,0.63}{{#1}}}
    \newcommand{\SpecialStringTok}[1]{\textcolor[rgb]{0.73,0.40,0.53}{{#1}}}
    \newcommand{\ImportTok}[1]{{#1}}
    \newcommand{\DocumentationTok}[1]{\textcolor[rgb]{0.73,0.13,0.13}{\textit{{#1}}}}
    \newcommand{\AnnotationTok}[1]{\textcolor[rgb]{0.38,0.63,0.69}{\textbf{\textit{{#1}}}}}
    \newcommand{\CommentVarTok}[1]{\textcolor[rgb]{0.38,0.63,0.69}{\textbf{\textit{{#1}}}}}
    \newcommand{\VariableTok}[1]{\textcolor[rgb]{0.10,0.09,0.49}{{#1}}}
    \newcommand{\ControlFlowTok}[1]{\textcolor[rgb]{0.00,0.44,0.13}{\textbf{{#1}}}}
    \newcommand{\OperatorTok}[1]{\textcolor[rgb]{0.40,0.40,0.40}{{#1}}}
    \newcommand{\BuiltInTok}[1]{{#1}}
    \newcommand{\ExtensionTok}[1]{{#1}}
    \newcommand{\PreprocessorTok}[1]{\textcolor[rgb]{0.74,0.48,0.00}{{#1}}}
    \newcommand{\AttributeTok}[1]{\textcolor[rgb]{0.49,0.56,0.16}{{#1}}}
    \newcommand{\InformationTok}[1]{\textcolor[rgb]{0.38,0.63,0.69}{\textbf{\textit{{#1}}}}}
    \newcommand{\WarningTok}[1]{\textcolor[rgb]{0.38,0.63,0.69}{\textbf{\textit{{#1}}}}}
    
    
    % Define a nice break command that doesn't care if a line doesn't already
    % exist.
    \def\br{\hspace*{\fill} \\* }
    % Math Jax compatability definitions
    \def\gt{>}
    \def\lt{<}
    % Document parameters
    \title{Lab 1}
    
    
    

    % Pygments definitions
    
\makeatletter
\def\PY@reset{\let\PY@it=\relax \let\PY@bf=\relax%
    \let\PY@ul=\relax \let\PY@tc=\relax%
    \let\PY@bc=\relax \let\PY@ff=\relax}
\def\PY@tok#1{\csname PY@tok@#1\endcsname}
\def\PY@toks#1+{\ifx\relax#1\empty\else%
    \PY@tok{#1}\expandafter\PY@toks\fi}
\def\PY@do#1{\PY@bc{\PY@tc{\PY@ul{%
    \PY@it{\PY@bf{\PY@ff{#1}}}}}}}
\def\PY#1#2{\PY@reset\PY@toks#1+\relax+\PY@do{#2}}

\expandafter\def\csname PY@tok@w\endcsname{\def\PY@tc##1{\textcolor[rgb]{0.73,0.73,0.73}{##1}}}
\expandafter\def\csname PY@tok@c\endcsname{\let\PY@it=\textit\def\PY@tc##1{\textcolor[rgb]{0.25,0.50,0.50}{##1}}}
\expandafter\def\csname PY@tok@cp\endcsname{\def\PY@tc##1{\textcolor[rgb]{0.74,0.48,0.00}{##1}}}
\expandafter\def\csname PY@tok@k\endcsname{\let\PY@bf=\textbf\def\PY@tc##1{\textcolor[rgb]{0.00,0.50,0.00}{##1}}}
\expandafter\def\csname PY@tok@kp\endcsname{\def\PY@tc##1{\textcolor[rgb]{0.00,0.50,0.00}{##1}}}
\expandafter\def\csname PY@tok@kt\endcsname{\def\PY@tc##1{\textcolor[rgb]{0.69,0.00,0.25}{##1}}}
\expandafter\def\csname PY@tok@o\endcsname{\def\PY@tc##1{\textcolor[rgb]{0.40,0.40,0.40}{##1}}}
\expandafter\def\csname PY@tok@ow\endcsname{\let\PY@bf=\textbf\def\PY@tc##1{\textcolor[rgb]{0.67,0.13,1.00}{##1}}}
\expandafter\def\csname PY@tok@nb\endcsname{\def\PY@tc##1{\textcolor[rgb]{0.00,0.50,0.00}{##1}}}
\expandafter\def\csname PY@tok@nf\endcsname{\def\PY@tc##1{\textcolor[rgb]{0.00,0.00,1.00}{##1}}}
\expandafter\def\csname PY@tok@nc\endcsname{\let\PY@bf=\textbf\def\PY@tc##1{\textcolor[rgb]{0.00,0.00,1.00}{##1}}}
\expandafter\def\csname PY@tok@nn\endcsname{\let\PY@bf=\textbf\def\PY@tc##1{\textcolor[rgb]{0.00,0.00,1.00}{##1}}}
\expandafter\def\csname PY@tok@ne\endcsname{\let\PY@bf=\textbf\def\PY@tc##1{\textcolor[rgb]{0.82,0.25,0.23}{##1}}}
\expandafter\def\csname PY@tok@nv\endcsname{\def\PY@tc##1{\textcolor[rgb]{0.10,0.09,0.49}{##1}}}
\expandafter\def\csname PY@tok@no\endcsname{\def\PY@tc##1{\textcolor[rgb]{0.53,0.00,0.00}{##1}}}
\expandafter\def\csname PY@tok@nl\endcsname{\def\PY@tc##1{\textcolor[rgb]{0.63,0.63,0.00}{##1}}}
\expandafter\def\csname PY@tok@ni\endcsname{\let\PY@bf=\textbf\def\PY@tc##1{\textcolor[rgb]{0.60,0.60,0.60}{##1}}}
\expandafter\def\csname PY@tok@na\endcsname{\def\PY@tc##1{\textcolor[rgb]{0.49,0.56,0.16}{##1}}}
\expandafter\def\csname PY@tok@nt\endcsname{\let\PY@bf=\textbf\def\PY@tc##1{\textcolor[rgb]{0.00,0.50,0.00}{##1}}}
\expandafter\def\csname PY@tok@nd\endcsname{\def\PY@tc##1{\textcolor[rgb]{0.67,0.13,1.00}{##1}}}
\expandafter\def\csname PY@tok@s\endcsname{\def\PY@tc##1{\textcolor[rgb]{0.73,0.13,0.13}{##1}}}
\expandafter\def\csname PY@tok@sd\endcsname{\let\PY@it=\textit\def\PY@tc##1{\textcolor[rgb]{0.73,0.13,0.13}{##1}}}
\expandafter\def\csname PY@tok@si\endcsname{\let\PY@bf=\textbf\def\PY@tc##1{\textcolor[rgb]{0.73,0.40,0.53}{##1}}}
\expandafter\def\csname PY@tok@se\endcsname{\let\PY@bf=\textbf\def\PY@tc##1{\textcolor[rgb]{0.73,0.40,0.13}{##1}}}
\expandafter\def\csname PY@tok@sr\endcsname{\def\PY@tc##1{\textcolor[rgb]{0.73,0.40,0.53}{##1}}}
\expandafter\def\csname PY@tok@ss\endcsname{\def\PY@tc##1{\textcolor[rgb]{0.10,0.09,0.49}{##1}}}
\expandafter\def\csname PY@tok@sx\endcsname{\def\PY@tc##1{\textcolor[rgb]{0.00,0.50,0.00}{##1}}}
\expandafter\def\csname PY@tok@m\endcsname{\def\PY@tc##1{\textcolor[rgb]{0.40,0.40,0.40}{##1}}}
\expandafter\def\csname PY@tok@gh\endcsname{\let\PY@bf=\textbf\def\PY@tc##1{\textcolor[rgb]{0.00,0.00,0.50}{##1}}}
\expandafter\def\csname PY@tok@gu\endcsname{\let\PY@bf=\textbf\def\PY@tc##1{\textcolor[rgb]{0.50,0.00,0.50}{##1}}}
\expandafter\def\csname PY@tok@gd\endcsname{\def\PY@tc##1{\textcolor[rgb]{0.63,0.00,0.00}{##1}}}
\expandafter\def\csname PY@tok@gi\endcsname{\def\PY@tc##1{\textcolor[rgb]{0.00,0.63,0.00}{##1}}}
\expandafter\def\csname PY@tok@gr\endcsname{\def\PY@tc##1{\textcolor[rgb]{1.00,0.00,0.00}{##1}}}
\expandafter\def\csname PY@tok@ge\endcsname{\let\PY@it=\textit}
\expandafter\def\csname PY@tok@gs\endcsname{\let\PY@bf=\textbf}
\expandafter\def\csname PY@tok@gp\endcsname{\let\PY@bf=\textbf\def\PY@tc##1{\textcolor[rgb]{0.00,0.00,0.50}{##1}}}
\expandafter\def\csname PY@tok@go\endcsname{\def\PY@tc##1{\textcolor[rgb]{0.53,0.53,0.53}{##1}}}
\expandafter\def\csname PY@tok@gt\endcsname{\def\PY@tc##1{\textcolor[rgb]{0.00,0.27,0.87}{##1}}}
\expandafter\def\csname PY@tok@err\endcsname{\def\PY@bc##1{\setlength{\fboxsep}{0pt}\fcolorbox[rgb]{1.00,0.00,0.00}{1,1,1}{\strut ##1}}}
\expandafter\def\csname PY@tok@kc\endcsname{\let\PY@bf=\textbf\def\PY@tc##1{\textcolor[rgb]{0.00,0.50,0.00}{##1}}}
\expandafter\def\csname PY@tok@kd\endcsname{\let\PY@bf=\textbf\def\PY@tc##1{\textcolor[rgb]{0.00,0.50,0.00}{##1}}}
\expandafter\def\csname PY@tok@kn\endcsname{\let\PY@bf=\textbf\def\PY@tc##1{\textcolor[rgb]{0.00,0.50,0.00}{##1}}}
\expandafter\def\csname PY@tok@kr\endcsname{\let\PY@bf=\textbf\def\PY@tc##1{\textcolor[rgb]{0.00,0.50,0.00}{##1}}}
\expandafter\def\csname PY@tok@bp\endcsname{\def\PY@tc##1{\textcolor[rgb]{0.00,0.50,0.00}{##1}}}
\expandafter\def\csname PY@tok@fm\endcsname{\def\PY@tc##1{\textcolor[rgb]{0.00,0.00,1.00}{##1}}}
\expandafter\def\csname PY@tok@vc\endcsname{\def\PY@tc##1{\textcolor[rgb]{0.10,0.09,0.49}{##1}}}
\expandafter\def\csname PY@tok@vg\endcsname{\def\PY@tc##1{\textcolor[rgb]{0.10,0.09,0.49}{##1}}}
\expandafter\def\csname PY@tok@vi\endcsname{\def\PY@tc##1{\textcolor[rgb]{0.10,0.09,0.49}{##1}}}
\expandafter\def\csname PY@tok@vm\endcsname{\def\PY@tc##1{\textcolor[rgb]{0.10,0.09,0.49}{##1}}}
\expandafter\def\csname PY@tok@sa\endcsname{\def\PY@tc##1{\textcolor[rgb]{0.73,0.13,0.13}{##1}}}
\expandafter\def\csname PY@tok@sb\endcsname{\def\PY@tc##1{\textcolor[rgb]{0.73,0.13,0.13}{##1}}}
\expandafter\def\csname PY@tok@sc\endcsname{\def\PY@tc##1{\textcolor[rgb]{0.73,0.13,0.13}{##1}}}
\expandafter\def\csname PY@tok@dl\endcsname{\def\PY@tc##1{\textcolor[rgb]{0.73,0.13,0.13}{##1}}}
\expandafter\def\csname PY@tok@s2\endcsname{\def\PY@tc##1{\textcolor[rgb]{0.73,0.13,0.13}{##1}}}
\expandafter\def\csname PY@tok@sh\endcsname{\def\PY@tc##1{\textcolor[rgb]{0.73,0.13,0.13}{##1}}}
\expandafter\def\csname PY@tok@s1\endcsname{\def\PY@tc##1{\textcolor[rgb]{0.73,0.13,0.13}{##1}}}
\expandafter\def\csname PY@tok@mb\endcsname{\def\PY@tc##1{\textcolor[rgb]{0.40,0.40,0.40}{##1}}}
\expandafter\def\csname PY@tok@mf\endcsname{\def\PY@tc##1{\textcolor[rgb]{0.40,0.40,0.40}{##1}}}
\expandafter\def\csname PY@tok@mh\endcsname{\def\PY@tc##1{\textcolor[rgb]{0.40,0.40,0.40}{##1}}}
\expandafter\def\csname PY@tok@mi\endcsname{\def\PY@tc##1{\textcolor[rgb]{0.40,0.40,0.40}{##1}}}
\expandafter\def\csname PY@tok@il\endcsname{\def\PY@tc##1{\textcolor[rgb]{0.40,0.40,0.40}{##1}}}
\expandafter\def\csname PY@tok@mo\endcsname{\def\PY@tc##1{\textcolor[rgb]{0.40,0.40,0.40}{##1}}}
\expandafter\def\csname PY@tok@ch\endcsname{\let\PY@it=\textit\def\PY@tc##1{\textcolor[rgb]{0.25,0.50,0.50}{##1}}}
\expandafter\def\csname PY@tok@cm\endcsname{\let\PY@it=\textit\def\PY@tc##1{\textcolor[rgb]{0.25,0.50,0.50}{##1}}}
\expandafter\def\csname PY@tok@cpf\endcsname{\let\PY@it=\textit\def\PY@tc##1{\textcolor[rgb]{0.25,0.50,0.50}{##1}}}
\expandafter\def\csname PY@tok@c1\endcsname{\let\PY@it=\textit\def\PY@tc##1{\textcolor[rgb]{0.25,0.50,0.50}{##1}}}
\expandafter\def\csname PY@tok@cs\endcsname{\let\PY@it=\textit\def\PY@tc##1{\textcolor[rgb]{0.25,0.50,0.50}{##1}}}

\def\PYZbs{\char`\\}
\def\PYZus{\char`\_}
\def\PYZob{\char`\{}
\def\PYZcb{\char`\}}
\def\PYZca{\char`\^}
\def\PYZam{\char`\&}
\def\PYZlt{\char`\<}
\def\PYZgt{\char`\>}
\def\PYZsh{\char`\#}
\def\PYZpc{\char`\%}
\def\PYZdl{\char`\$}
\def\PYZhy{\char`\-}
\def\PYZsq{\char`\'}
\def\PYZdq{\char`\"}
\def\PYZti{\char`\~}
% for compatibility with earlier versions
\def\PYZat{@}
\def\PYZlb{[}
\def\PYZrb{]}
\makeatother


    % Exact colors from NB
    \definecolor{incolor}{rgb}{0.0, 0.0, 0.5}
    \definecolor{outcolor}{rgb}{0.545, 0.0, 0.0}



    
    % Prevent overflowing lines due to hard-to-break entities
    \sloppy 
    % Setup hyperref package
    \hypersetup{
      breaklinks=true,  % so long urls are correctly broken across lines
      colorlinks=true,
      urlcolor=urlcolor,
      linkcolor=linkcolor,
      citecolor=citecolor,
      }
    % Slightly bigger margins than the latex defaults
    
    \geometry{verbose,tmargin=1in,bmargin=1in,lmargin=1in,rmargin=1in}
    
    

    \begin{document}
    
    
    \maketitle
    
    

    
    \begin{Verbatim}[commandchars=\\\{\}]
{\color{incolor}In [{\color{incolor}131}]:} \PY{c+c1}{\PYZsh{} Import and initialize Matlab\PYZsq{}s Python API because lab1 is Matlab\PYZhy{}specific.}
          \PY{c+c1}{\PYZsh{} NOTE: this takes a while (also I/O blocking)}
          \PY{k+kn}{import} \PY{n+nn}{matlab}\PY{n+nn}{.}\PY{n+nn}{engine}
          \PY{n}{eng} \PY{o}{=} \PY{n}{matlab}\PY{o}{.}\PY{n}{engine}\PY{o}{.}\PY{n}{start\PYZus{}matlab}\PY{p}{(}\PY{p}{)}
\end{Verbatim}


    \begin{Verbatim}[commandchars=\\\{\}]
{\color{incolor}In [{\color{incolor}132}]:} \PY{c+c1}{\PYZsh{} Load python dependencies}
          \PY{k+kn}{import} \PY{n+nn}{sys}
          \PY{k+kn}{import} \PY{n+nn}{matplotlib}\PY{n+nn}{.}\PY{n+nn}{pyplot} \PY{k}{as} \PY{n+nn}{plt}
          \PY{k+kn}{import} \PY{n+nn}{matplotlib}\PY{n+nn}{.}\PY{n+nn}{image} \PY{k}{as} \PY{n+nn}{mpimg}
          \PY{k+kn}{import} \PY{n+nn}{numpy} \PY{k}{as} \PY{n+nn}{np}
\end{Verbatim}


    \begin{Verbatim}[commandchars=\\\{\}]
{\color{incolor}In [{\color{incolor}133}]:} \PY{c+c1}{\PYZsh{} Redirect Matlab stdout and stderr to buffers}
          
          \PY{k+kn}{import} \PY{n+nn}{io}
          \PY{n}{out} \PY{o}{=} \PY{n}{io}\PY{o}{.}\PY{n}{StringIO}\PY{p}{(}\PY{p}{)}
          \PY{n}{err} \PY{o}{=} \PY{n}{io}\PY{o}{.}\PY{n}{StringIO}\PY{p}{(}\PY{p}{)}
          
          \PY{k}{def} \PY{n+nf}{clean\PYZus{}buffers}\PY{p}{(}\PY{n}{out}\PY{p}{,} \PY{n}{err}\PY{p}{)}\PY{p}{:}
              \PY{n}{out}\PY{o}{.}\PY{n}{seek}\PY{p}{(}\PY{l+m+mi}{0}\PY{p}{)}
              \PY{n}{out}\PY{o}{.}\PY{n}{truncate}\PY{p}{(}\PY{l+m+mi}{0}\PY{p}{)}
              
              \PY{n}{err}\PY{o}{.}\PY{n}{seek}\PY{p}{(}\PY{l+m+mi}{0}\PY{p}{)}
              \PY{n}{err}\PY{o}{.}\PY{n}{truncate}\PY{p}{(}\PY{l+m+mi}{0}\PY{p}{)}
\end{Verbatim}


    \begin{enumerate}
\def\labelenumi{\arabic{enumi}.}
\tightlist
\item
  Image Classes
\end{enumerate}

1(a) Load a truecolor image of your choice into a matrix I into Matlab
\% using imread.

    \begin{Verbatim}[commandchars=\\\{\}]
{\color{incolor}In [{\color{incolor}134}]:} \PY{n}{eng}\PY{o}{.}\PY{n}{eval}\PY{p}{(}\PY{l+s+s1}{\PYZsq{}}\PY{l+s+s1}{I = imread(}\PY{l+s+se}{\PYZbs{}\PYZsq{}}\PY{l+s+s1}{pamukkale.jpg}\PY{l+s+se}{\PYZbs{}\PYZsq{}}\PY{l+s+s1}{);}\PY{l+s+s1}{\PYZsq{}}\PY{p}{,} \PY{n}{nargout}\PY{o}{=}\PY{l+m+mi}{0}\PY{p}{,} \PY{n}{stdout}\PY{o}{=}\PY{n}{out}\PY{p}{,} \PY{n}{stderr}\PY{o}{=}\PY{n}{err}\PY{p}{)}
          \PY{n}{eng}\PY{o}{.}\PY{n}{eval}\PY{p}{(}\PY{l+s+s1}{\PYZsq{}}\PY{l+s+s1}{figure; image(I); truesize;}\PY{l+s+s1}{\PYZsq{}}\PY{p}{,} \PY{n}{nargout}\PY{o}{=}\PY{l+m+mi}{0}\PY{p}{,} \PY{n}{stdout}\PY{o}{=}\PY{n}{out}\PY{p}{,} \PY{n}{stderr}\PY{o}{=}\PY{n}{err}\PY{p}{)}
          \PY{n}{eng}\PY{o}{.}\PY{n}{eval}\PY{p}{(}\PY{l+s+s1}{\PYZsq{}}\PY{l+s+s1}{set(gca,}\PY{l+s+se}{\PYZbs{}\PYZsq{}}\PY{l+s+s1}{position}\PY{l+s+se}{\PYZbs{}\PYZsq{}}\PY{l+s+s1}{,[0 0 1 1],}\PY{l+s+se}{\PYZbs{}\PYZsq{}}\PY{l+s+s1}{units}\PY{l+s+se}{\PYZbs{}\PYZsq{}}\PY{l+s+s1}{,}\PY{l+s+se}{\PYZbs{}\PYZsq{}}\PY{l+s+s1}{normalized}\PY{l+s+se}{\PYZbs{}\PYZsq{}}\PY{l+s+s1}{, }\PY{l+s+se}{\PYZbs{}\PYZsq{}}\PY{l+s+s1}{Visible}\PY{l+s+se}{\PYZbs{}\PYZsq{}}\PY{l+s+s1}{, }\PY{l+s+se}{\PYZbs{}\PYZsq{}}\PY{l+s+s1}{off}\PY{l+s+se}{\PYZbs{}\PYZsq{}}\PY{l+s+s1}{);}\PY{l+s+s1}{\PYZsq{}}\PY{p}{,} \PY{n}{stdout}\PY{o}{=}\PY{n}{out}\PY{p}{,} \PY{n}{stderr}\PY{o}{=}\PY{n}{err}\PY{p}{)}
          \PY{n}{img} \PY{o}{=} \PY{n}{mpimg}\PY{o}{.}\PY{n}{imread}\PY{p}{(}\PY{l+s+s1}{\PYZsq{}}\PY{l+s+s1}{1a\PYZus{}pamukkale.bmp}\PY{l+s+s1}{\PYZsq{}}\PY{p}{)}
          \PY{n}{plt}\PY{o}{.}\PY{n}{imshow}\PY{p}{(}\PY{n}{img}\PY{p}{)}
          \PY{n}{plt}\PY{o}{.}\PY{n}{title}\PY{p}{(}\PY{l+s+s1}{\PYZsq{}}\PY{l+s+s1}{1(a) Pamukkale, Turkey. ©Paul Williams 2017}\PY{l+s+s1}{\PYZsq{}}\PY{p}{)}
          \PY{n}{plt}\PY{o}{.}\PY{n}{show}\PY{p}{(}\PY{p}{)}
          
          \PY{c+c1}{\PYZsh{} What\PYZsq{}s the image\PYZsq{}s class?}
          \PY{n}{clean\PYZus{}buffers}\PY{p}{(}\PY{n}{out}\PY{p}{,} \PY{n}{err}\PY{p}{)}
          \PY{n}{eng}\PY{o}{.}\PY{n}{eval}\PY{p}{(}\PY{l+s+s1}{\PYZsq{}}\PY{l+s+s1}{fprintf(}\PY{l+s+se}{\PYZbs{}\PYZsq{}}\PY{l+s+s1}{1(a) class of I is }\PY{l+s+si}{\PYZpc{}s}\PY{l+s+se}{\PYZbs{}\PYZbs{}}\PY{l+s+s1}{n}\PY{l+s+se}{\PYZbs{}\PYZsq{}}\PY{l+s+s1}{, class(I));}\PY{l+s+s1}{\PYZsq{}}\PY{p}{,} \PY{n}{nargout}\PY{o}{=}\PY{l+m+mi}{0}\PY{p}{,} \PY{n}{stdout}\PY{o}{=}\PY{n}{out}\PY{p}{,} \PY{n}{stderr}\PY{o}{=}\PY{n}{err}\PY{p}{)}
          \PY{n}{sys}\PY{o}{.}\PY{n}{stdout}\PY{o}{.}\PY{n}{write}\PY{p}{(}\PY{n}{out}\PY{o}{.}\PY{n}{getvalue}\PY{p}{(}\PY{p}{)}\PY{p}{)}
          
          \PY{c+c1}{\PYZsh{} What\PYZsq{}s its size?}
          \PY{n}{clean\PYZus{}buffers}\PY{p}{(}\PY{n}{out}\PY{p}{,} \PY{n}{err}\PY{p}{)}
          \PY{n}{eng}\PY{o}{.}\PY{n}{eval}\PY{p}{(}\PY{l+s+s1}{\PYZsq{}}\PY{l+s+s1}{fprintf(}\PY{l+s+se}{\PYZbs{}\PYZsq{}}\PY{l+s+s1}{1(a) size of I is }\PY{l+s+si}{\PYZpc{}s}\PY{l+s+se}{\PYZbs{}\PYZbs{}}\PY{l+s+s1}{n}\PY{l+s+se}{\PYZbs{}\PYZsq{}}\PY{l+s+s1}{, mat2str(size(I)));}\PY{l+s+s1}{\PYZsq{}}\PY{p}{,} \PY{n}{nargout}\PY{o}{=}\PY{l+m+mi}{0}\PY{p}{,} \PY{n}{stdout}\PY{o}{=}\PY{n}{out}\PY{p}{,} \PY{n}{stderr}\PY{o}{=}\PY{n}{err}\PY{p}{)}
          \PY{n}{sys}\PY{o}{.}\PY{n}{stdout}\PY{o}{.}\PY{n}{write}\PY{p}{(}\PY{n}{out}\PY{o}{.}\PY{n}{getvalue}\PY{p}{(}\PY{p}{)}\PY{p}{)}
          
          
          \PY{c+c1}{\PYZsh{} How many bands does it have?}
          \PY{n}{clean\PYZus{}buffers}\PY{p}{(}\PY{n}{out}\PY{p}{,} \PY{n}{err}\PY{p}{)}
          \PY{n}{eng}\PY{o}{.}\PY{n}{eval}\PY{p}{(}\PY{l+s+s1}{\PYZsq{}}\PY{l+s+s1}{fprintf(}\PY{l+s+se}{\PYZbs{}\PYZsq{}}\PY{l+s+s1}{1(a) the number of bands of I is }\PY{l+s+si}{\PYZpc{}d}\PY{l+s+se}{\PYZbs{}\PYZbs{}}\PY{l+s+s1}{n}\PY{l+s+se}{\PYZbs{}\PYZsq{}}\PY{l+s+s1}{, size(I, 3));}\PY{l+s+s1}{\PYZsq{}}\PY{p}{,} \PY{n}{nargout}\PY{o}{=}\PY{l+m+mi}{0}\PY{p}{,} \PY{n}{stdout}\PY{o}{=}\PY{n}{out}\PY{p}{,} \PY{n}{stderr}\PY{o}{=}\PY{n}{err}\PY{p}{)}
          \PY{n}{sys}\PY{o}{.}\PY{n}{stdout}\PY{o}{.}\PY{n}{write}\PY{p}{(}\PY{n}{out}\PY{o}{.}\PY{n}{getvalue}\PY{p}{(}\PY{p}{)}\PY{p}{)}
          
          
          \PY{c+c1}{\PYZsh{} Find its minimum and maximum values using}
          \PY{n}{clean\PYZus{}buffers}\PY{p}{(}\PY{n}{out}\PY{p}{,} \PY{n}{err}\PY{p}{)}
          \PY{n}{eng}\PY{o}{.}\PY{n}{eval}\PY{p}{(}\PY{l+s+s1}{\PYZsq{}}\PY{l+s+s1}{fprintf(}\PY{l+s+se}{\PYZbs{}\PYZsq{}}\PY{l+s+s1}{1(a) min of I is }\PY{l+s+si}{\PYZpc{}d}\PY{l+s+se}{\PYZbs{}\PYZbs{}}\PY{l+s+s1}{n}\PY{l+s+se}{\PYZbs{}\PYZsq{}}\PY{l+s+s1}{, min(I(:)));}\PY{l+s+s1}{\PYZsq{}}\PY{p}{,} \PY{n}{nargout}\PY{o}{=}\PY{l+m+mi}{0}\PY{p}{,} \PY{n}{stdout}\PY{o}{=}\PY{n}{out}\PY{p}{,} \PY{n}{stderr}\PY{o}{=}\PY{n}{err}\PY{p}{)}
          \PY{n}{eng}\PY{o}{.}\PY{n}{eval}\PY{p}{(}\PY{l+s+s1}{\PYZsq{}}\PY{l+s+s1}{fprintf(}\PY{l+s+se}{\PYZbs{}\PYZsq{}}\PY{l+s+s1}{1(a) max of I is }\PY{l+s+si}{\PYZpc{}d}\PY{l+s+se}{\PYZbs{}\PYZbs{}}\PY{l+s+s1}{n}\PY{l+s+se}{\PYZbs{}\PYZsq{}}\PY{l+s+s1}{, max(I(:)));}\PY{l+s+s1}{\PYZsq{}}\PY{p}{,} \PY{n}{nargout}\PY{o}{=}\PY{l+m+mi}{0}\PY{p}{,} \PY{n}{stdout}\PY{o}{=}\PY{n}{out}\PY{p}{,} \PY{n}{stderr}\PY{o}{=}\PY{n}{err}\PY{p}{)}
          \PY{n}{sys}\PY{o}{.}\PY{n}{stdout}\PY{o}{.}\PY{n}{write}\PY{p}{(}\PY{n}{out}\PY{o}{.}\PY{n}{getvalue}\PY{p}{(}\PY{p}{)}\PY{p}{)}
\end{Verbatim}


    \begin{center}
    \adjustimage{max size={0.9\linewidth}{0.9\paperheight}}{output_4_0.png}
    \end{center}
    { \hspace*{\fill} \\}
    
    \begin{Verbatim}[commandchars=\\\{\}]
1(a) class of I is uint8
1(a) size of I is [536 858 3]
1(a) the number of bands of I is 3
1(a) min of I is 0
1(a) max of I is 255

    \end{Verbatim}

    1(b)

Convert the image into a 64-bit floating-point number array using the
command

    \begin{Verbatim}[commandchars=\\\{\}]
{\color{incolor}In [{\color{incolor}143}]:} \PY{n}{clean\PYZus{}buffers}\PY{p}{(}\PY{n}{out}\PY{p}{,} \PY{n}{err}\PY{p}{)}
          \PY{n}{eng}\PY{o}{.}\PY{n}{eval}\PY{p}{(}\PY{l+s+s1}{\PYZsq{}}\PY{l+s+s1}{D = double(I);}\PY{l+s+s1}{\PYZsq{}}\PY{p}{,} \PY{n}{nargout}\PY{o}{=}\PY{l+m+mi}{0}\PY{p}{,} \PY{n}{stdout}\PY{o}{=}\PY{n}{out}\PY{p}{,} \PY{n}{stderr}\PY{o}{=}\PY{n}{err}\PY{p}{)}
          
          \PY{c+c1}{\PYZsh{} and check its class.}
          \PY{n}{clean\PYZus{}buffers}\PY{p}{(}\PY{n}{out}\PY{p}{,} \PY{n}{err}\PY{p}{)}
          \PY{n}{eng}\PY{o}{.}\PY{n}{eval}\PY{p}{(}\PY{l+s+s1}{\PYZsq{}}\PY{l+s+s1}{fprintf(}\PY{l+s+se}{\PYZbs{}\PYZsq{}}\PY{l+s+s1}{1(b) class of D is }\PY{l+s+si}{\PYZpc{}s}\PY{l+s+se}{\PYZbs{}\PYZbs{}}\PY{l+s+s1}{n}\PY{l+s+se}{\PYZbs{}\PYZsq{}}\PY{l+s+s1}{, class(D));}\PY{l+s+s1}{\PYZsq{}}\PY{p}{,} \PY{n}{nargout}\PY{o}{=}\PY{l+m+mi}{0}\PY{p}{,} \PY{n}{stdout}\PY{o}{=}\PY{n}{out}\PY{p}{,} \PY{n}{stderr}\PY{o}{=}\PY{n}{err}\PY{p}{)}
          \PY{n}{sys}\PY{o}{.}\PY{n}{stdout}\PY{o}{.}\PY{n}{write}\PY{p}{(}\PY{n}{out}\PY{o}{.}\PY{n}{getvalue}\PY{p}{(}\PY{p}{)}\PY{p}{)}
          
          \PY{c+c1}{\PYZsh{} Find the maximum and minimum values of D.}
          \PY{n}{clean\PYZus{}buffers}\PY{p}{(}\PY{n}{out}\PY{p}{,} \PY{n}{err}\PY{p}{)}
          \PY{n}{eng}\PY{o}{.}\PY{n}{eval}\PY{p}{(}\PY{l+s+s1}{\PYZsq{}}\PY{l+s+s1}{fprintf(}\PY{l+s+se}{\PYZbs{}\PYZsq{}}\PY{l+s+s1}{1(b) min o D is }\PY{l+s+si}{\PYZpc{}d}\PY{l+s+se}{\PYZbs{}\PYZbs{}}\PY{l+s+s1}{n}\PY{l+s+se}{\PYZbs{}\PYZsq{}}\PY{l+s+s1}{, min(D(:)));}\PY{l+s+s1}{\PYZsq{}}\PY{p}{,} \PY{n}{nargout}\PY{o}{=}\PY{l+m+mi}{0}\PY{p}{,} \PY{n}{stdout}\PY{o}{=}\PY{n}{out}\PY{p}{,} \PY{n}{stderr}\PY{o}{=}\PY{n}{err}\PY{p}{)}
          \PY{n}{eng}\PY{o}{.}\PY{n}{eval}\PY{p}{(}\PY{l+s+s1}{\PYZsq{}}\PY{l+s+s1}{fprintf(}\PY{l+s+se}{\PYZbs{}\PYZsq{}}\PY{l+s+s1}{1(b) max of D is }\PY{l+s+si}{\PYZpc{}d}\PY{l+s+se}{\PYZbs{}\PYZbs{}}\PY{l+s+s1}{n}\PY{l+s+se}{\PYZbs{}\PYZsq{}}\PY{l+s+s1}{, max(D(:)));}\PY{l+s+s1}{\PYZsq{}}\PY{p}{,} \PY{n}{nargout}\PY{o}{=}\PY{l+m+mi}{0}\PY{p}{,} \PY{n}{stdout}\PY{o}{=}\PY{n}{out}\PY{p}{,} \PY{n}{stderr}\PY{o}{=}\PY{n}{err}\PY{p}{)}
          \PY{n}{sys}\PY{o}{.}\PY{n}{stdout}\PY{o}{.}\PY{n}{write}\PY{p}{(}\PY{n}{out}\PY{o}{.}\PY{n}{getvalue}\PY{p}{(}\PY{p}{)}\PY{p}{)}
\end{Verbatim}


    \begin{Verbatim}[commandchars=\\\{\}]
1(b) class of D is double
1(b) min o D is 0
1(b) max of D is 255

    \end{Verbatim}

    1(c)

Now display D using

    \begin{Verbatim}[commandchars=\\\{\}]
{\color{incolor}In [{\color{incolor}144}]:} \PY{n}{eng}\PY{o}{.}\PY{n}{eval}\PY{p}{(}\PY{l+s+s1}{\PYZsq{}}\PY{l+s+s1}{figure; image(D); truesize;}\PY{l+s+s1}{\PYZsq{}}\PY{p}{,} \PY{n}{nargout}\PY{o}{=}\PY{l+m+mi}{0}\PY{p}{,} \PY{n}{stdout}\PY{o}{=}\PY{n}{out}\PY{p}{,} \PY{n}{stderr}\PY{o}{=}\PY{n}{err}\PY{p}{)}
          \PY{n}{eng}\PY{o}{.}\PY{n}{eval}\PY{p}{(}\PY{l+s+s1}{\PYZsq{}}\PY{l+s+s1}{set(gca,}\PY{l+s+se}{\PYZbs{}\PYZsq{}}\PY{l+s+s1}{position}\PY{l+s+se}{\PYZbs{}\PYZsq{}}\PY{l+s+s1}{,[0 0 1 1],}\PY{l+s+se}{\PYZbs{}\PYZsq{}}\PY{l+s+s1}{units}\PY{l+s+se}{\PYZbs{}\PYZsq{}}\PY{l+s+s1}{,}\PY{l+s+se}{\PYZbs{}\PYZsq{}}\PY{l+s+s1}{normalized}\PY{l+s+se}{\PYZbs{}\PYZsq{}}\PY{l+s+s1}{, }\PY{l+s+se}{\PYZbs{}\PYZsq{}}\PY{l+s+s1}{Visible}\PY{l+s+se}{\PYZbs{}\PYZsq{}}\PY{l+s+s1}{, }\PY{l+s+se}{\PYZbs{}\PYZsq{}}\PY{l+s+s1}{off}\PY{l+s+se}{\PYZbs{}\PYZsq{}}\PY{l+s+s1}{);}\PY{l+s+s1}{\PYZsq{}}\PY{p}{,} \PY{n}{stdout}\PY{o}{=}\PY{n}{out}\PY{p}{,} \PY{n}{stderr}\PY{o}{=}\PY{n}{err}\PY{p}{)}
          \PY{n}{img} \PY{o}{=} \PY{n}{mpimg}\PY{o}{.}\PY{n}{imread}\PY{p}{(}\PY{l+s+s1}{\PYZsq{}}\PY{l+s+s1}{1c\PYZus{}doubled.bmp}\PY{l+s+s1}{\PYZsq{}}\PY{p}{)}
          \PY{n}{plt}\PY{o}{.}\PY{n}{imshow}\PY{p}{(}\PY{n}{img}\PY{p}{)}
          \PY{n}{plt}\PY{o}{.}\PY{n}{title}\PY{p}{(}\PY{l+s+s1}{\PYZsq{}}\PY{l+s+s1}{1(c) Doubled}\PY{l+s+s1}{\PYZsq{}}\PY{p}{)}
          \PY{n}{plt}\PY{o}{.}\PY{n}{show}\PY{p}{(}\PY{p}{)}
\end{Verbatim}


    \begin{center}
    \adjustimage{max size={0.9\linewidth}{0.9\paperheight}}{output_8_0.png}
    \end{center}
    { \hspace*{\fill} \\}
    
    1(c)

What happens? Read the Matlab help pages on images as necessary to
understand what's going on.

\hypertarget{answer-the-image-appears-almost-all-white-with-some-cyan.-behind-the-scenes-when-the-image-class-is-double-matlab-only-accepts-float-values-between-0-and-1.-any---in-this-case-almost-all---values-above-1.0-are-set-to-1.-the-reason-that-the-image-displayed-have-mostly-cyan-is-that-in-those-pixel-locations-the-red-values-were-0-to-start-with-while-the-blue-and-green-were-over-1-and-thus-set-to-1.-everywhere-else-the-values-have-been-set-to-1-1-1-thus-appearing-white.}{%
\paragraph{Answer: The image appears almost all white with some cyan.
Behind the scenes, when the image class is double, Matlab only accepts
float values between 0 and 1. Any - in this case almost all - values
above 1.0 are set to 1. The reason that the image displayed have mostly
cyan is that in those pixel locations, the red values were 0 to start
with, while the blue and green were over 1 and thus set to 1. Everywhere
else, the values have been set to (1, 1, 1) thus appearing
white.}\label{answer-the-image-appears-almost-all-white-with-some-cyan.-behind-the-scenes-when-the-image-class-is-double-matlab-only-accepts-float-values-between-0-and-1.-any---in-this-case-almost-all---values-above-1.0-are-set-to-1.-the-reason-that-the-image-displayed-have-mostly-cyan-is-that-in-those-pixel-locations-the-red-values-were-0-to-start-with-while-the-blue-and-green-were-over-1-and-thus-set-to-1.-everywhere-else-the-values-have-been-set-to-1-1-1-thus-appearing-white.}}

    1(c)

What arithmetic operation must you perform on D to make it displayable
with image(D)? That is, replace D with an arithmetically altered version
of D - i.e.~D = f(D) where f() s some arithmetic operation like
multiplicatin or division by a constant. te that as you experiment with
this, if you replace D with the wrong thing, you will have to execute D
= double(I); to get it back.

\hypertarget{answer-d-d255}{%
\paragraph{Answer: D = D/255}\label{answer-d-d255}}

    \begin{Verbatim}[commandchars=\\\{\}]
{\color{incolor}In [{\color{incolor}145}]:} \PY{n}{eng}\PY{o}{.}\PY{n}{eval}\PY{p}{(}\PY{l+s+s1}{\PYZsq{}}\PY{l+s+s1}{D = D/255;}\PY{l+s+s1}{\PYZsq{}}\PY{p}{,} \PY{n}{nargout}\PY{o}{=}\PY{l+m+mi}{0}\PY{p}{,} \PY{n}{stdout}\PY{o}{=}\PY{n}{out}\PY{p}{,} \PY{n}{stderr}\PY{o}{=}\PY{n}{err}\PY{p}{)}
\end{Verbatim}


    \begin{Verbatim}[commandchars=\\\{\}]
{\color{incolor}In [{\color{incolor}146}]:} \PY{c+c1}{\PYZsh{} Find the maximum and minimum values of your altered D.}
          \PY{n}{clean\PYZus{}buffers}\PY{p}{(}\PY{n}{out}\PY{p}{,} \PY{n}{err}\PY{p}{)}
          \PY{n}{eng}\PY{o}{.}\PY{n}{eval}\PY{p}{(}\PY{l+s+s1}{\PYZsq{}}\PY{l+s+s1}{fprintf(}\PY{l+s+se}{\PYZbs{}\PYZsq{}}\PY{l+s+s1}{1(c) min o corrected D is }\PY{l+s+si}{\PYZpc{}d}\PY{l+s+se}{\PYZbs{}\PYZbs{}}\PY{l+s+s1}{n}\PY{l+s+se}{\PYZbs{}\PYZsq{}}\PY{l+s+s1}{, min(D(:)));}\PY{l+s+s1}{\PYZsq{}}\PY{p}{,} \PY{n}{nargout}\PY{o}{=}\PY{l+m+mi}{0}\PY{p}{,} \PY{n}{stdout}\PY{o}{=}\PY{n}{out}\PY{p}{,} \PY{n}{stderr}\PY{o}{=}\PY{n}{err}\PY{p}{)}
          \PY{n}{eng}\PY{o}{.}\PY{n}{eval}\PY{p}{(}\PY{l+s+s1}{\PYZsq{}}\PY{l+s+s1}{fprintf(}\PY{l+s+se}{\PYZbs{}\PYZsq{}}\PY{l+s+s1}{1(c) max of corrected D is }\PY{l+s+si}{\PYZpc{}d}\PY{l+s+se}{\PYZbs{}\PYZbs{}}\PY{l+s+s1}{n}\PY{l+s+se}{\PYZbs{}\PYZsq{}}\PY{l+s+s1}{, max(D(:)));}\PY{l+s+s1}{\PYZsq{}}\PY{p}{,} \PY{n}{nargout}\PY{o}{=}\PY{l+m+mi}{0}\PY{p}{,} \PY{n}{stdout}\PY{o}{=}\PY{n}{out}\PY{p}{,} \PY{n}{stderr}\PY{o}{=}\PY{n}{err}\PY{p}{)}
          \PY{n}{eng}\PY{o}{.}\PY{n}{eval}\PY{p}{(}\PY{l+s+s1}{\PYZsq{}}\PY{l+s+s1}{fprintf(}\PY{l+s+se}{\PYZbs{}\PYZsq{}}\PY{l+s+s1}{1(c) class of corrected D is }\PY{l+s+si}{\PYZpc{}s}\PY{l+s+se}{\PYZbs{}\PYZbs{}}\PY{l+s+s1}{n}\PY{l+s+se}{\PYZbs{}\PYZsq{}}\PY{l+s+s1}{, class(D));}\PY{l+s+s1}{\PYZsq{}}\PY{p}{,} \PY{n}{nargout}\PY{o}{=}\PY{l+m+mi}{0}\PY{p}{,} \PY{n}{stdout}\PY{o}{=}\PY{n}{out}\PY{p}{,} \PY{n}{stderr}\PY{o}{=}\PY{n}{err}\PY{p}{)}
          \PY{n}{eng}\PY{o}{.}\PY{n}{eval}\PY{p}{(}\PY{l+s+s1}{\PYZsq{}}\PY{l+s+s1}{fprintf(}\PY{l+s+se}{\PYZbs{}\PYZsq{}}\PY{l+s+s1}{1(c) size of corrected D is }\PY{l+s+si}{\PYZpc{}s}\PY{l+s+se}{\PYZbs{}\PYZbs{}}\PY{l+s+s1}{n}\PY{l+s+se}{\PYZbs{}\PYZsq{}}\PY{l+s+s1}{, mat2str(size(D)));}\PY{l+s+s1}{\PYZsq{}}\PY{p}{,} \PY{n}{nargout}\PY{o}{=}\PY{l+m+mi}{0}\PY{p}{,} \PY{n}{stdout}\PY{o}{=}\PY{n}{out}\PY{p}{,} \PY{n}{stderr}\PY{o}{=}\PY{n}{err}\PY{p}{)}
          \PY{n}{sys}\PY{o}{.}\PY{n}{stdout}\PY{o}{.}\PY{n}{write}\PY{p}{(}\PY{n}{out}\PY{o}{.}\PY{n}{getvalue}\PY{p}{(}\PY{p}{)}\PY{p}{)}
\end{Verbatim}


    \begin{Verbatim}[commandchars=\\\{\}]
1(c) min o corrected D is 0
1(c) max of corrected D is 1
1(c) class of corrected D is double
1(c) size of corrected D is [536 858 3]

    \end{Verbatim}

    1(d)

Once you are able to run

    \begin{Verbatim}[commandchars=\\\{\}]
{\color{incolor}In [{\color{incolor}151}]:} \PY{n}{eng}\PY{o}{.}\PY{n}{eval}\PY{p}{(}\PY{l+s+s1}{\PYZsq{}}\PY{l+s+s1}{figure; image(D); truesize;}\PY{l+s+s1}{\PYZsq{}}\PY{p}{,} \PY{n}{nargout}\PY{o}{=}\PY{l+m+mi}{0}\PY{p}{,} \PY{n}{stdout}\PY{o}{=}\PY{n}{out}\PY{p}{,} \PY{n}{stderr}\PY{o}{=}\PY{n}{err}\PY{p}{)}
          \PY{n}{eng}\PY{o}{.}\PY{n}{eval}\PY{p}{(}\PY{l+s+s1}{\PYZsq{}}\PY{l+s+s1}{set(gca,}\PY{l+s+se}{\PYZbs{}\PYZsq{}}\PY{l+s+s1}{position}\PY{l+s+se}{\PYZbs{}\PYZsq{}}\PY{l+s+s1}{,[0 0 1 1],}\PY{l+s+se}{\PYZbs{}\PYZsq{}}\PY{l+s+s1}{units}\PY{l+s+se}{\PYZbs{}\PYZsq{}}\PY{l+s+s1}{,}\PY{l+s+se}{\PYZbs{}\PYZsq{}}\PY{l+s+s1}{normalized}\PY{l+s+se}{\PYZbs{}\PYZsq{}}\PY{l+s+s1}{, }\PY{l+s+se}{\PYZbs{}\PYZsq{}}\PY{l+s+s1}{Visible}\PY{l+s+se}{\PYZbs{}\PYZsq{}}\PY{l+s+s1}{, }\PY{l+s+se}{\PYZbs{}\PYZsq{}}\PY{l+s+s1}{off}\PY{l+s+se}{\PYZbs{}\PYZsq{}}\PY{l+s+s1}{);}\PY{l+s+s1}{\PYZsq{}}\PY{p}{,} \PY{n}{stdout}\PY{o}{=}\PY{n}{out}\PY{p}{,} \PY{n}{stderr}\PY{o}{=}\PY{n}{err}\PY{p}{)}
          \PY{n}{img} \PY{o}{=} \PY{n}{mpimg}\PY{o}{.}\PY{n}{imread}\PY{p}{(}\PY{l+s+s1}{\PYZsq{}}\PY{l+s+s1}{1d\PYZus{}doubled\PYZus{}scaled.bmp}\PY{l+s+s1}{\PYZsq{}}\PY{p}{)}
          \PY{n}{plt}\PY{o}{.}\PY{n}{imshow}\PY{p}{(}\PY{n}{img}\PY{p}{)}
          \PY{n}{plt}\PY{o}{.}\PY{n}{title}\PY{p}{(}\PY{l+s+s1}{\PYZsq{}}\PY{l+s+s1}{1(d) Doubled but Scaled}\PY{l+s+s1}{\PYZsq{}}\PY{p}{)}
          \PY{n}{plt}\PY{o}{.}\PY{n}{show}\PY{p}{(}\PY{p}{)}
          
          \PY{c+c1}{\PYZsh{} and get an image that looks correct, convert D back to 8 bits per pixel by entering}
          \PY{n}{eng}\PY{o}{.}\PY{n}{eval}\PY{p}{(}\PY{l+s+s1}{\PYZsq{}}\PY{l+s+s1}{U8 = uint8(D);}\PY{l+s+s1}{\PYZsq{}}\PY{p}{,} \PY{n}{nargout}\PY{o}{=}\PY{l+m+mi}{0}\PY{p}{,} \PY{n}{stdout}\PY{o}{=}\PY{n}{out}\PY{p}{,} \PY{n}{stderr}\PY{o}{=}\PY{n}{err}\PY{p}{)}
          \PY{n}{eng}\PY{o}{.}\PY{n}{eval}\PY{p}{(}\PY{l+s+s1}{\PYZsq{}}\PY{l+s+s1}{figure; image(U8); truesize;}\PY{l+s+s1}{\PYZsq{}}\PY{p}{,} \PY{n}{nargout}\PY{o}{=}\PY{l+m+mi}{0}\PY{p}{,} \PY{n}{stdout}\PY{o}{=}\PY{n}{out}\PY{p}{,} \PY{n}{stderr}\PY{o}{=}\PY{n}{err}\PY{p}{)}
          \PY{n}{eng}\PY{o}{.}\PY{n}{eval}\PY{p}{(}\PY{l+s+s1}{\PYZsq{}}\PY{l+s+s1}{set(gca,}\PY{l+s+se}{\PYZbs{}\PYZsq{}}\PY{l+s+s1}{position}\PY{l+s+se}{\PYZbs{}\PYZsq{}}\PY{l+s+s1}{,[0 0 1 1],}\PY{l+s+se}{\PYZbs{}\PYZsq{}}\PY{l+s+s1}{units}\PY{l+s+se}{\PYZbs{}\PYZsq{}}\PY{l+s+s1}{,}\PY{l+s+se}{\PYZbs{}\PYZsq{}}\PY{l+s+s1}{normalized}\PY{l+s+se}{\PYZbs{}\PYZsq{}}\PY{l+s+s1}{, }\PY{l+s+se}{\PYZbs{}\PYZsq{}}\PY{l+s+s1}{Visible}\PY{l+s+se}{\PYZbs{}\PYZsq{}}\PY{l+s+s1}{, }\PY{l+s+se}{\PYZbs{}\PYZsq{}}\PY{l+s+s1}{off}\PY{l+s+se}{\PYZbs{}\PYZsq{}}\PY{l+s+s1}{);}\PY{l+s+s1}{\PYZsq{}}\PY{p}{,} \PY{n}{stdout}\PY{o}{=}\PY{n}{out}\PY{p}{,} \PY{n}{stderr}\PY{o}{=}\PY{n}{err}\PY{p}{)}
          \PY{n}{img} \PY{o}{=} \PY{n}{mpimg}\PY{o}{.}\PY{n}{imread}\PY{p}{(}\PY{l+s+s1}{\PYZsq{}}\PY{l+s+s1}{1d\PYZus{}inted.bmp}\PY{l+s+s1}{\PYZsq{}}\PY{p}{)}
          \PY{n}{plt}\PY{o}{.}\PY{n}{figure}\PY{p}{(}\PY{p}{)}
          \PY{n}{plt}\PY{o}{.}\PY{n}{imshow}\PY{p}{(}\PY{n}{img}\PY{p}{)}
          \PY{n}{plt}\PY{o}{.}\PY{n}{title}\PY{p}{(}\PY{l+s+s1}{\PYZsq{}}\PY{l+s+s1}{1(d) Inted}\PY{l+s+s1}{\PYZsq{}}\PY{p}{)}
          \PY{n}{plt}\PY{o}{.}\PY{n}{show}\PY{p}{(}\PY{p}{)}
\end{Verbatim}


    \begin{center}
    \adjustimage{max size={0.9\linewidth}{0.9\paperheight}}{output_14_0.png}
    \end{center}
    { \hspace*{\fill} \\}
    
    \begin{center}
    \adjustimage{max size={0.9\linewidth}{0.9\paperheight}}{output_14_1.png}
    \end{center}
    { \hspace*{\fill} \\}
    
    What do you see?

\hypertarget{answer-the-displayed-image-seems-all-black.}{%
\paragraph{Answer: The displayed image seems all
black.}\label{answer-the-displayed-image-seems-all-black.}}

    1(d) Find the maximum and minimum values of U8.

    \begin{Verbatim}[commandchars=\\\{\}]
{\color{incolor}In [{\color{incolor}152}]:} \PY{n}{clean\PYZus{}buffers}\PY{p}{(}\PY{n}{out}\PY{p}{,} \PY{n}{err}\PY{p}{)}
          \PY{n}{eng}\PY{o}{.}\PY{n}{eval}\PY{p}{(}\PY{l+s+s1}{\PYZsq{}}\PY{l+s+s1}{fprintf(}\PY{l+s+se}{\PYZbs{}\PYZsq{}}\PY{l+s+s1}{1(d) min of U8 is }\PY{l+s+si}{\PYZpc{}d}\PY{l+s+se}{\PYZbs{}\PYZbs{}}\PY{l+s+s1}{n}\PY{l+s+se}{\PYZbs{}\PYZsq{}}\PY{l+s+s1}{, min(U8(:)));}\PY{l+s+s1}{\PYZsq{}}\PY{p}{,} \PY{n}{nargout}\PY{o}{=}\PY{l+m+mi}{0}\PY{p}{,} \PY{n}{stdout}\PY{o}{=}\PY{n}{out}\PY{p}{,} \PY{n}{stderr}\PY{o}{=}\PY{n}{err}\PY{p}{)}
          \PY{n}{eng}\PY{o}{.}\PY{n}{eval}\PY{p}{(}\PY{l+s+s1}{\PYZsq{}}\PY{l+s+s1}{fprintf(}\PY{l+s+se}{\PYZbs{}\PYZsq{}}\PY{l+s+s1}{1(d) max of U8 is }\PY{l+s+si}{\PYZpc{}d}\PY{l+s+se}{\PYZbs{}\PYZbs{}}\PY{l+s+s1}{n}\PY{l+s+se}{\PYZbs{}\PYZsq{}}\PY{l+s+s1}{, max(U8(:)));}\PY{l+s+s1}{\PYZsq{}}\PY{p}{,} \PY{n}{nargout}\PY{o}{=}\PY{l+m+mi}{0}\PY{p}{,} \PY{n}{stdout}\PY{o}{=}\PY{n}{out}\PY{p}{,} \PY{n}{stderr}\PY{o}{=}\PY{n}{err}\PY{p}{)}
          \PY{n}{sys}\PY{o}{.}\PY{n}{stdout}\PY{o}{.}\PY{n}{write}\PY{p}{(}\PY{n}{out}\PY{o}{.}\PY{n}{getvalue}\PY{p}{(}\PY{p}{)}\PY{p}{)}
\end{Verbatim}


    \begin{Verbatim}[commandchars=\\\{\}]
1(d) min of U8 is 0
1(d) max of U8 is 1

    \end{Verbatim}

    1(d) What has the conversion done?

\hypertarget{answer-the-conversion-rounded-the-floats-between-0-and-1-to-integers-0-and-1.-the-image-remains-8-bit-with-3-bands-so-matlab-treats-it-as-if-it-were-still-a-color-image-with-a-range-of-integers-0-255.-however-1-looks-similar-to-0-so-that-the-image-looks-all-black-though-it-isnt.}{%
\paragraph{Answer: The conversion rounded the floats between 0 and 1 to
integers 0 and 1. The image remains 8-bit (with 3 bands) so Matlab
treats it as if it were still a color image with a range of (integers)
{[}0, 255{]}. However, 1 looks similar to 0 so that the image looks all
black though it
isn't.}\label{answer-the-conversion-rounded-the-floats-between-0-and-1-to-integers-0-and-1.-the-image-remains-8-bit-with-3-bands-so-matlab-treats-it-as-if-it-were-still-a-color-image-with-a-range-of-integers-0-255.-however-1-looks-similar-to-0-so-that-the-image-looks-all-black-though-it-isnt.}}

    1(d) Now do this:

    \begin{Verbatim}[commandchars=\\\{\}]
{\color{incolor}In [{\color{incolor}154}]:} \PY{n}{eng}\PY{o}{.}\PY{n}{eval}\PY{p}{(}\PY{l+s+s1}{\PYZsq{}}\PY{l+s+s1}{figure; image(U8*255); truesize;}\PY{l+s+s1}{\PYZsq{}}\PY{p}{,} \PY{n}{nargout}\PY{o}{=}\PY{l+m+mi}{0}\PY{p}{,} \PY{n}{stdout}\PY{o}{=}\PY{n}{out}\PY{p}{,} \PY{n}{stderr}\PY{o}{=}\PY{n}{err}\PY{p}{)}
          \PY{n}{eng}\PY{o}{.}\PY{n}{eval}\PY{p}{(}\PY{l+s+s1}{\PYZsq{}}\PY{l+s+s1}{set(gca,}\PY{l+s+se}{\PYZbs{}\PYZsq{}}\PY{l+s+s1}{position}\PY{l+s+se}{\PYZbs{}\PYZsq{}}\PY{l+s+s1}{,[0 0 1 1],}\PY{l+s+se}{\PYZbs{}\PYZsq{}}\PY{l+s+s1}{units}\PY{l+s+se}{\PYZbs{}\PYZsq{}}\PY{l+s+s1}{,}\PY{l+s+se}{\PYZbs{}\PYZsq{}}\PY{l+s+s1}{normalized}\PY{l+s+se}{\PYZbs{}\PYZsq{}}\PY{l+s+s1}{, }\PY{l+s+se}{\PYZbs{}\PYZsq{}}\PY{l+s+s1}{Visible}\PY{l+s+se}{\PYZbs{}\PYZsq{}}\PY{l+s+s1}{, }\PY{l+s+se}{\PYZbs{}\PYZsq{}}\PY{l+s+s1}{off}\PY{l+s+se}{\PYZbs{}\PYZsq{}}\PY{l+s+s1}{);}\PY{l+s+s1}{\PYZsq{}}\PY{p}{,} \PY{n}{stdout}\PY{o}{=}\PY{n}{out}\PY{p}{,} \PY{n}{stderr}\PY{o}{=}\PY{n}{err}\PY{p}{)}
          \PY{n}{img} \PY{o}{=} \PY{n}{mpimg}\PY{o}{.}\PY{n}{imread}\PY{p}{(}\PY{l+s+s1}{\PYZsq{}}\PY{l+s+s1}{1d\PYZus{}inted\PYZus{}255.bmp}\PY{l+s+s1}{\PYZsq{}}\PY{p}{)}
          \PY{n}{plt}\PY{o}{.}\PY{n}{figure}\PY{p}{(}\PY{p}{)}
          \PY{n}{plt}\PY{o}{.}\PY{n}{imshow}\PY{p}{(}\PY{n}{img}\PY{p}{)}
          \PY{n}{plt}\PY{o}{.}\PY{n}{title}\PY{p}{(}\PY{l+s+s1}{\PYZsq{}}\PY{l+s+s1}{1(d) Inted to 255}\PY{l+s+s1}{\PYZsq{}}\PY{p}{)}
          \PY{n}{plt}\PY{o}{.}\PY{n}{show}\PY{p}{(}\PY{p}{)}
\end{Verbatim}


    \begin{center}
    \adjustimage{max size={0.9\linewidth}{0.9\paperheight}}{output_20_0.png}
    \end{center}
    { \hspace*{\fill} \\}
    
    1(d)

What if any colors do you see?

\hypertarget{answer-blue-green-red-yellow-cyan-magenta-white-black.}{%
\paragraph{Answer: Blue, green, red, yellow, cyan, magenta, white,
black.}\label{answer-blue-green-red-yellow-cyan-magenta-white-black.}}

Why would a binary image have more colors than just black and white?

\hypertarget{answer-different-combinations-23-8-of-0s-and-255s-in-three-color-locations-can-still-produce-rgb-and-their-basic-combinations-cyan-magenta-yellow-plus-white-and-black.}{%
\paragraph{Answer: Different combinations (2\^{}3 = 8) of 0s and 255s in
three color locations can still produce RGB and their basic combinations
(cyan, magenta, yellow), plus white and
black.}\label{answer-different-combinations-23-8-of-0s-and-255s-in-three-color-locations-can-still-produce-rgb-and-their-basic-combinations-cyan-magenta-yellow-plus-white-and-black.}}

    1(e)

In your own words, write out a short explanation of the differences
between images of class uint8 and double.

\hypertarget{answer-double-images-are-more-accomodating-in-representating-uint8-and-uint16-etc.-each-pixel-location-needs-to-have-a-float-value-between-0-and-1.-in-contrast-uint8-images-are-specific-and-have-integer-values-between-0-and-255.}{%
\paragraph{Answer: double images are more accomodating in representating
uint8 and uint16, etc. Each pixel location needs to have a float value
between 0 and 1. In contrast, uint8 images are specific and have integer
values between 0 and
255.}\label{answer-double-images-are-more-accomodating-in-representating-uint8-and-uint16-etc.-each-pixel-location-needs-to-have-a-float-value-between-0-and-1.-in-contrast-uint8-images-are-specific-and-have-integer-values-between-0-and-255.}}

Explain what it takes to convert between the formats without losing the
intensities in the images.

\hypertarget{answer-in-order-to-convert-uint8-intensity-images-to-double-the-integers-first-need-to-be-converted-to-double.-then-they-need-to-be-scaled-by-the-max-value-255.-in-reverse-double-images-need-to-be-scaled-up-and-then-rounded-to-0-255.}{%
\paragraph{Answer: In order to convert uint8 intensity images to double,
the integers first need to be converted to double. Then they need to be
scaled by the max value (255). In reverse, double images need to be
scaled up and then rounded to {[}0,
255{]}.}\label{answer-in-order-to-convert-uint8-intensity-images-to-double-the-integers-first-need-to-be-converted-to-double.-then-they-need-to-be-scaled-by-the-max-value-255.-in-reverse-double-images-need-to-be-scaled-up-and-then-rounded-to-0-255.}}

    \begin{enumerate}
\def\labelenumi{\arabic{enumi}.}
\setcounter{enumi}{1}
\tightlist
\item
  Loading, displaying, and saving color mapped (aka indexed) images.
\end{enumerate}

2(a) Load chweel\_mapped.gif and its color map into a matrix in Matlab.

    \begin{Verbatim}[commandchars=\\\{\}]
{\color{incolor}In [{\color{incolor}155}]:} \PY{n}{eng}\PY{o}{.}\PY{n}{eval}\PY{p}{(}\PY{l+s+s1}{\PYZsq{}}\PY{l+s+s1}{[gif cmap\PYZus{}gif] = imread(}\PY{l+s+se}{\PYZbs{}\PYZsq{}}\PY{l+s+s1}{cwheel\PYZus{}mapped.gif}\PY{l+s+se}{\PYZbs{}\PYZsq{}}\PY{l+s+s1}{);}\PY{l+s+s1}{\PYZsq{}}\PY{p}{,} \PY{n}{nargout}\PY{o}{=}\PY{l+m+mi}{0}\PY{p}{,} \PY{n}{stdout}\PY{o}{=}\PY{n}{out}\PY{p}{,} \PY{n}{stderr}\PY{o}{=}\PY{n}{err}\PY{p}{)}
\end{Verbatim}


    2(a) Use the class function to determine the type of the image

    \begin{Verbatim}[commandchars=\\\{\}]
{\color{incolor}In [{\color{incolor}156}]:} \PY{n}{clean\PYZus{}buffers}\PY{p}{(}\PY{n}{out}\PY{p}{,} \PY{n}{err}\PY{p}{)}
          \PY{n}{eng}\PY{o}{.}\PY{n}{eval}\PY{p}{(}\PY{l+s+s1}{\PYZsq{}}\PY{l+s+s1}{fprintf(}\PY{l+s+se}{\PYZbs{}\PYZsq{}}\PY{l+s+s1}{2(a) class of gif is }\PY{l+s+si}{\PYZpc{}s}\PY{l+s+se}{\PYZbs{}\PYZbs{}}\PY{l+s+s1}{n}\PY{l+s+se}{\PYZbs{}\PYZsq{}}\PY{l+s+s1}{, class(gif));}\PY{l+s+s1}{\PYZsq{}}\PY{p}{,} \PY{n}{nargout}\PY{o}{=}\PY{l+m+mi}{0}\PY{p}{,} \PY{n}{stdout}\PY{o}{=}\PY{n}{out}\PY{p}{,} \PY{n}{stderr}\PY{o}{=}\PY{n}{err}\PY{p}{)}
          \PY{n}{sys}\PY{o}{.}\PY{n}{stdout}\PY{o}{.}\PY{n}{write}\PY{p}{(}\PY{n}{out}\PY{o}{.}\PY{n}{getvalue}\PY{p}{(}\PY{p}{)}\PY{p}{)}
\end{Verbatim}


    \begin{Verbatim}[commandchars=\\\{\}]
2(a) class of gif is uint8

    \end{Verbatim}

    2(a) Use size to determine its dimensions

    \begin{Verbatim}[commandchars=\\\{\}]
{\color{incolor}In [{\color{incolor}157}]:} \PY{n}{clean\PYZus{}buffers}\PY{p}{(}\PY{n}{out}\PY{p}{,} \PY{n}{err}\PY{p}{)}
          \PY{n}{eng}\PY{o}{.}\PY{n}{eval}\PY{p}{(}\PY{l+s+s1}{\PYZsq{}}\PY{l+s+s1}{fprintf(}\PY{l+s+se}{\PYZbs{}\PYZsq{}}\PY{l+s+s1}{2(a) size of gif is }\PY{l+s+si}{\PYZpc{}s}\PY{l+s+se}{\PYZbs{}\PYZbs{}}\PY{l+s+s1}{n}\PY{l+s+se}{\PYZbs{}\PYZsq{}}\PY{l+s+s1}{, mat2str(size(gif)));}\PY{l+s+s1}{\PYZsq{}}\PY{p}{,} \PY{n}{nargout}\PY{o}{=}\PY{l+m+mi}{0}\PY{p}{,} \PY{n}{stdout}\PY{o}{=}\PY{n}{out}\PY{p}{,} \PY{n}{stderr}\PY{o}{=}\PY{n}{err}\PY{p}{)}
          \PY{n}{sys}\PY{o}{.}\PY{n}{stdout}\PY{o}{.}\PY{n}{write}\PY{p}{(}\PY{n}{out}\PY{o}{.}\PY{n}{getvalue}\PY{p}{(}\PY{p}{)}\PY{p}{)}
\end{Verbatim}


    \begin{Verbatim}[commandchars=\\\{\}]
2(a) size of gif is [600 800]

    \end{Verbatim}

    2(a)

Also, note the file size in bytes of cwheel\_mapped.gif. How does this
compare to the dimensions and file size of cwheel.bmp Answer: bmp is
1.4MB, but git is 137KB.

imread returns two data structures. The first one is a matrix containing
the image. The second is the image?s color map. Think of the color map
as a palette of colors arranged in a N � 3 table, where N is the
number of colors. The value at a pixel in the image is not an intensity,
but an index into the color map ? the number of the row that contains
the color vector for that pixel.

How many colors are in the palette for cwheel\_mapped.gif? Note:
Typically all x-bit images have 256 colors.

    \begin{Verbatim}[commandchars=\\\{\}]
{\color{incolor}In [{\color{incolor}158}]:} \PY{n}{clean\PYZus{}buffers}\PY{p}{(}\PY{n}{out}\PY{p}{,} \PY{n}{err}\PY{p}{)}
          \PY{n}{eng}\PY{o}{.}\PY{n}{eval}\PY{p}{(}\PY{l+s+s1}{\PYZsq{}}\PY{l+s+s1}{fprintf(}\PY{l+s+se}{\PYZbs{}\PYZsq{}}\PY{l+s+s1}{2(a) There are }\PY{l+s+si}{\PYZpc{}d}\PY{l+s+s1}{ colors in the palette for cwheel\PYZus{}mapped.gif}\PY{l+s+se}{\PYZbs{}\PYZbs{}}\PY{l+s+s1}{n}\PY{l+s+se}{\PYZbs{}\PYZsq{}}\PY{l+s+s1}{, size(cmap\PYZus{}gif, 1));}\PY{l+s+s1}{\PYZsq{}}\PY{p}{,} \PY{n}{nargout}\PY{o}{=}\PY{l+m+mi}{0}\PY{p}{,} \PY{n}{stdout}\PY{o}{=}\PY{n}{out}\PY{p}{,} \PY{n}{stderr}\PY{o}{=}\PY{n}{err}\PY{p}{)}
          \PY{n}{sys}\PY{o}{.}\PY{n}{stdout}\PY{o}{.}\PY{n}{write}\PY{p}{(}\PY{n}{out}\PY{o}{.}\PY{n}{getvalue}\PY{p}{(}\PY{p}{)}\PY{p}{)}
\end{Verbatim}


    \begin{Verbatim}[commandchars=\\\{\}]
2(a) There are 256 colors in the palette for cwheel\_mapped.gif

    \end{Verbatim}

    2(b)

Use

    \begin{Verbatim}[commandchars=\\\{\}]
{\color{incolor}In [{\color{incolor}160}]:} \PY{n}{eng}\PY{o}{.}\PY{n}{eval}\PY{p}{(}\PY{l+s+s1}{\PYZsq{}}\PY{l+s+s1}{figure; image(gif); truesize;}\PY{l+s+s1}{\PYZsq{}}\PY{p}{,} \PY{n}{nargout}\PY{o}{=}\PY{l+m+mi}{0}\PY{p}{,} \PY{n}{stdout}\PY{o}{=}\PY{n}{out}\PY{p}{,} \PY{n}{stderr}\PY{o}{=}\PY{n}{err}\PY{p}{)}
          \PY{n}{eng}\PY{o}{.}\PY{n}{eval}\PY{p}{(}\PY{l+s+s1}{\PYZsq{}}\PY{l+s+s1}{set(gca,}\PY{l+s+se}{\PYZbs{}\PYZsq{}}\PY{l+s+s1}{position}\PY{l+s+se}{\PYZbs{}\PYZsq{}}\PY{l+s+s1}{,[0 0 1 1],}\PY{l+s+se}{\PYZbs{}\PYZsq{}}\PY{l+s+s1}{units}\PY{l+s+se}{\PYZbs{}\PYZsq{}}\PY{l+s+s1}{,}\PY{l+s+se}{\PYZbs{}\PYZsq{}}\PY{l+s+s1}{normalized}\PY{l+s+se}{\PYZbs{}\PYZsq{}}\PY{l+s+s1}{, }\PY{l+s+se}{\PYZbs{}\PYZsq{}}\PY{l+s+s1}{Visible}\PY{l+s+se}{\PYZbs{}\PYZsq{}}\PY{l+s+s1}{, }\PY{l+s+se}{\PYZbs{}\PYZsq{}}\PY{l+s+s1}{off}\PY{l+s+se}{\PYZbs{}\PYZsq{}}\PY{l+s+s1}{);}\PY{l+s+s1}{\PYZsq{}}\PY{p}{,} \PY{n}{stdout}\PY{o}{=}\PY{n}{out}\PY{p}{,} \PY{n}{stderr}\PY{o}{=}\PY{n}{err}\PY{p}{)}
          \PY{n}{img} \PY{o}{=} \PY{n}{mpimg}\PY{o}{.}\PY{n}{imread}\PY{p}{(}\PY{l+s+s1}{\PYZsq{}}\PY{l+s+s1}{2b\PYZus{}gif\PYZus{}yellow.bmp}\PY{l+s+s1}{\PYZsq{}}\PY{p}{)}
          \PY{n}{plt}\PY{o}{.}\PY{n}{imshow}\PY{p}{(}\PY{n}{img}\PY{p}{)}
          \PY{n}{plt}\PY{o}{.}\PY{n}{title}\PY{p}{(}\PY{l+s+s1}{\PYZsq{}}\PY{l+s+s1}{2(b)}\PY{l+s+s1}{\PYZsq{}}\PY{p}{)}
          \PY{n}{plt}\PY{o}{.}\PY{n}{show}\PY{p}{(}\PY{p}{)}
\end{Verbatim}


    \begin{center}
    \adjustimage{max size={0.9\linewidth}{0.9\paperheight}}{output_32_0.png}
    \end{center}
    { \hspace*{\fill} \\}
    
    2(b)

What do you see? Describe its appearance. \#\#\#\# Answer: Answer: It's
desplayed incorrectly - mostly yellow and blue.

Now use the command to associate with the new figure the color map
loaded along with cwheel\_mapped.gif.

    \begin{Verbatim}[commandchars=\\\{\}]
{\color{incolor}In [{\color{incolor}161}]:} \PY{n}{eng}\PY{o}{.}\PY{n}{eval}\PY{p}{(}\PY{l+s+s1}{\PYZsq{}}\PY{l+s+s1}{colormap(cmap\PYZus{}gif); title(}\PY{l+s+se}{\PYZbs{}\PYZsq{}}\PY{l+s+s1}{gif}\PY{l+s+se}{\PYZbs{}\PYZsq{}}\PY{l+s+s1}{);}\PY{l+s+s1}{\PYZsq{}}\PY{p}{,} \PY{n}{nargout}\PY{o}{=}\PY{l+m+mi}{0}\PY{p}{,} \PY{n}{stdout}\PY{o}{=}\PY{n}{out}\PY{p}{,} \PY{n}{stderr}\PY{o}{=}\PY{n}{err}\PY{p}{)}
          \PY{n}{eng}\PY{o}{.}\PY{n}{eval}\PY{p}{(}\PY{l+s+s1}{\PYZsq{}}\PY{l+s+s1}{set(gca,}\PY{l+s+se}{\PYZbs{}\PYZsq{}}\PY{l+s+s1}{position}\PY{l+s+se}{\PYZbs{}\PYZsq{}}\PY{l+s+s1}{,[0 0 1 1],}\PY{l+s+se}{\PYZbs{}\PYZsq{}}\PY{l+s+s1}{units}\PY{l+s+se}{\PYZbs{}\PYZsq{}}\PY{l+s+s1}{,}\PY{l+s+se}{\PYZbs{}\PYZsq{}}\PY{l+s+s1}{normalized}\PY{l+s+se}{\PYZbs{}\PYZsq{}}\PY{l+s+s1}{, }\PY{l+s+se}{\PYZbs{}\PYZsq{}}\PY{l+s+s1}{Visible}\PY{l+s+se}{\PYZbs{}\PYZsq{}}\PY{l+s+s1}{, }\PY{l+s+se}{\PYZbs{}\PYZsq{}}\PY{l+s+s1}{off}\PY{l+s+se}{\PYZbs{}\PYZsq{}}\PY{l+s+s1}{);}\PY{l+s+s1}{\PYZsq{}}\PY{p}{,} \PY{n}{stdout}\PY{o}{=}\PY{n}{out}\PY{p}{,} \PY{n}{stderr}\PY{o}{=}\PY{n}{err}\PY{p}{)}
          \PY{n}{img} \PY{o}{=} \PY{n}{mpimg}\PY{o}{.}\PY{n}{imread}\PY{p}{(}\PY{l+s+s1}{\PYZsq{}}\PY{l+s+s1}{2b\PYZus{}gif\PYZus{}color.bmp}\PY{l+s+s1}{\PYZsq{}}\PY{p}{)}
          \PY{n}{plt}\PY{o}{.}\PY{n}{imshow}\PY{p}{(}\PY{n}{img}\PY{p}{)}
          \PY{n}{plt}\PY{o}{.}\PY{n}{title}\PY{p}{(}\PY{l+s+s1}{\PYZsq{}}\PY{l+s+s1}{2(b)}\PY{l+s+s1}{\PYZsq{}}\PY{p}{)}
          \PY{n}{plt}\PY{o}{.}\PY{n}{show}\PY{p}{(}\PY{p}{)}
\end{Verbatim}


    \begin{center}
    \adjustimage{max size={0.9\linewidth}{0.9\paperheight}}{output_34_0.png}
    \end{center}
    { \hspace*{\fill} \\}
    
    2(b)

Now what do you see? How does it compare to the first figure you
generated in part 2a?

\hypertarget{answer-the-image-is-displayed-correctly-using-the-colormap-compared-to-the-gif.}{%
\paragraph{Answer: The image is displayed correctly using the colormap
compared to the
gif.}\label{answer-the-image-is-displayed-correctly-using-the-colormap-compared-to-the-gif.}}

    2(c)

Create a new color map, say cmap2, from the old one, cmap, as follows

    \begin{Verbatim}[commandchars=\\\{\}]
{\color{incolor}In [{\color{incolor}162}]:} \PY{n}{eng}\PY{o}{.}\PY{n}{eval}\PY{p}{(}\PY{l+s+s1}{\PYZsq{}}\PY{l+s+s1}{temp = mean(cmap\PYZus{}gif,2);}\PY{l+s+s1}{\PYZsq{}}\PY{p}{,} \PY{n}{nargout}\PY{o}{=}\PY{l+m+mi}{0}\PY{p}{,} \PY{n}{stdout}\PY{o}{=}\PY{n}{out}\PY{p}{,} \PY{n}{stderr}\PY{o}{=}\PY{n}{err}\PY{p}{)}
          \PY{n}{eng}\PY{o}{.}\PY{n}{eval}\PY{p}{(}\PY{l+s+s1}{\PYZsq{}}\PY{l+s+s1}{cmap2 = [temp temp temp];}\PY{l+s+s1}{\PYZsq{}}\PY{p}{,} \PY{n}{nargout}\PY{o}{=}\PY{l+m+mi}{0}\PY{p}{,} \PY{n}{stdout}\PY{o}{=}\PY{n}{out}\PY{p}{,} \PY{n}{stderr}\PY{o}{=}\PY{n}{err}\PY{p}{)}
\end{Verbatim}


    2(c)

What did this just do?

\hypertarget{answer-the-first-step-took-the-mean-among-the-3-bands-of-each-color-in-the-palette-of-the-image-and-constructed-another-palette-setting-3-bands-to-the-average-color-intensity-of-the-original-palette.-if-the-three-bands-are-the-same-its-just-grey-by-definition.}{%
\paragraph{Answer: The first step took the mean among the 3 bands of
each color in the palette of the image and constructed another palette
setting 3 bands to the average color intensity of the original palette.
If the three bands are the same, it's just grey by
definition.}\label{answer-the-first-step-took-the-mean-among-the-3-bands-of-each-color-in-the-palette-of-the-image-and-constructed-another-palette-setting-3-bands-to-the-average-color-intensity-of-the-original-palette.-if-the-three-bands-are-the-same-its-just-grey-by-definition.}}

    2(c)

Select the figure that displays cwheel\_mapped.gif to which you did not
did not apply a color map, and apply the new color map using

    \begin{Verbatim}[commandchars=\\\{\}]
{\color{incolor}In [{\color{incolor}163}]:} \PY{n}{eng}\PY{o}{.}\PY{n}{eval}\PY{p}{(}\PY{l+s+s1}{\PYZsq{}}\PY{l+s+s1}{colormap(cmap2);}\PY{l+s+s1}{\PYZsq{}}\PY{p}{,} \PY{n}{nargout}\PY{o}{=}\PY{l+m+mi}{0}\PY{p}{,} \PY{n}{stdout}\PY{o}{=}\PY{n}{out}\PY{p}{,} \PY{n}{stderr}\PY{o}{=}\PY{n}{err}\PY{p}{)}
          \PY{n}{img} \PY{o}{=} \PY{n}{mpimg}\PY{o}{.}\PY{n}{imread}\PY{p}{(}\PY{l+s+s1}{\PYZsq{}}\PY{l+s+s1}{2b\PYZus{}gif\PYZus{}grey.bmp}\PY{l+s+s1}{\PYZsq{}}\PY{p}{)}
          \PY{n}{plt}\PY{o}{.}\PY{n}{imshow}\PY{p}{(}\PY{n}{img}\PY{p}{)}
          \PY{n}{plt}\PY{o}{.}\PY{n}{show}\PY{p}{(}\PY{p}{)}
\end{Verbatim}


    \begin{center}
    \adjustimage{max size={0.9\linewidth}{0.9\paperheight}}{output_40_0.png}
    \end{center}
    { \hspace*{\fill} \\}
    
    Now what do you see?

\hypertarget{answer-the-gif-has-been-converted-to-grayscale.}{%
\paragraph{Answer: The gif has been converted to
grayscale.}\label{answer-the-gif-has-been-converted-to-grayscale.}}

    2(d)

Now you are going to compare the 24-bit image, cwheel.bmp, with its
color mapped version, cwheel\_mapped.gif, displayed with its original
color map ? the one you loaded along with the image in part 2a.

    \begin{Verbatim}[commandchars=\\\{\}]
{\color{incolor}In [{\color{incolor}164}]:} \PY{n}{eng}\PY{o}{.}\PY{n}{eval}\PY{p}{(}\PY{l+s+s1}{\PYZsq{}}\PY{l+s+s1}{[bmp cmap\PYZus{}bmp] = imread(}\PY{l+s+se}{\PYZbs{}\PYZsq{}}\PY{l+s+s1}{cwheel.bmp}\PY{l+s+se}{\PYZbs{}\PYZsq{}}\PY{l+s+s1}{);}\PY{l+s+s1}{\PYZsq{}}\PY{p}{,} \PY{n}{nargout}\PY{o}{=}\PY{l+m+mi}{0}\PY{p}{,} \PY{n}{stdout}\PY{o}{=}\PY{n}{out}\PY{p}{,} \PY{n}{stderr}\PY{o}{=}\PY{n}{err}\PY{p}{)}
\end{Verbatim}


    2(d)

Describe the differences in appearance between cwheel.bmp and
cwheel\_mapped.gif.

\hypertarget{answer-cwheel.bmp-seems-much-smoother-in-color-transition-than-cwheel_mapped.}{%
\paragraph{Answer: cwheel.bmp seems much smoother in color transition
than
cwheel\_mapped.}\label{answer-cwheel.bmp-seems-much-smoother-in-color-transition-than-cwheel_mapped.}}

    \begin{Verbatim}[commandchars=\\\{\}]
{\color{incolor}In [{\color{incolor}63}]:} \PY{n}{eng}\PY{o}{.}\PY{n}{eval}\PY{p}{(}\PY{l+s+s1}{\PYZsq{}}\PY{l+s+s1}{figure; image(bmp); truesize; title(}\PY{l+s+se}{\PYZbs{}\PYZsq{}}\PY{l+s+s1}{bmp}\PY{l+s+se}{\PYZbs{}\PYZsq{}}\PY{l+s+s1}{);}\PY{l+s+s1}{\PYZsq{}}\PY{p}{,} \PY{n}{nargout}\PY{o}{=}\PY{l+m+mi}{0}\PY{p}{,} \PY{n}{stdout}\PY{o}{=}\PY{n}{out}\PY{p}{,} \PY{n}{stderr}\PY{o}{=}\PY{n}{err}\PY{p}{)}
         \PY{n}{eng}\PY{o}{.}\PY{n}{eval}\PY{p}{(}\PY{l+s+s1}{\PYZsq{}}\PY{l+s+s1}{set(gca,}\PY{l+s+se}{\PYZbs{}\PYZsq{}}\PY{l+s+s1}{position}\PY{l+s+se}{\PYZbs{}\PYZsq{}}\PY{l+s+s1}{,[0 0 1 1],}\PY{l+s+se}{\PYZbs{}\PYZsq{}}\PY{l+s+s1}{units}\PY{l+s+se}{\PYZbs{}\PYZsq{}}\PY{l+s+s1}{,}\PY{l+s+se}{\PYZbs{}\PYZsq{}}\PY{l+s+s1}{normalized}\PY{l+s+se}{\PYZbs{}\PYZsq{}}\PY{l+s+s1}{, }\PY{l+s+se}{\PYZbs{}\PYZsq{}}\PY{l+s+s1}{Visible}\PY{l+s+se}{\PYZbs{}\PYZsq{}}\PY{l+s+s1}{, }\PY{l+s+se}{\PYZbs{}\PYZsq{}}\PY{l+s+s1}{off}\PY{l+s+se}{\PYZbs{}\PYZsq{}}\PY{l+s+s1}{);}\PY{l+s+s1}{\PYZsq{}}\PY{p}{,} \PY{n}{nargout}\PY{o}{=}\PY{l+m+mi}{0}\PY{p}{,} \PY{n}{stdout}\PY{o}{=}\PY{n}{out}\PY{p}{,} \PY{n}{stderr}\PY{o}{=}\PY{n}{err}\PY{p}{)}
         \PY{n}{img} \PY{o}{=} \PY{n}{mpimg}\PY{o}{.}\PY{n}{imread}\PY{p}{(}\PY{l+s+s1}{\PYZsq{}}\PY{l+s+s1}{2b\PYZus{}gif\PYZus{}color\PYZus{}bmp.bmp}\PY{l+s+s1}{\PYZsq{}}\PY{p}{)}
         \PY{n}{plt}\PY{o}{.}\PY{n}{imshow}\PY{p}{(}\PY{n}{img}\PY{p}{)}
         \PY{n}{plt}\PY{o}{.}\PY{n}{show}\PY{p}{(}\PY{p}{)}
\end{Verbatim}


    \begin{center}
    \adjustimage{max size={0.9\linewidth}{0.9\paperheight}}{output_45_0.png}
    \end{center}
    { \hspace*{\fill} \\}
    
    2(d)

Based on their relative appearances, describe how gradual changes in
color are implemented (or approximated) by the color mapped image.

\hypertarget{answer-colormapped-indexed-images-are-based-on-the-true-color-images-but-binned-into-a-histogram-of-256-colors-instead-of-2563-possible-color-combinations-in-the-truecolor-image.-there-is-then-a-lookup-table-for-the-color-mapping.-the-noise-in-the-gif-seems-to-stem-from-cutoffs-from-adjacent-color-bins.}{%
\paragraph{Answer: Colormapped (``Indexed'') images are based on the
true color images but binned into a histogram of 256 colors instead of
256\^{}3 possible color combinations in the truecolor image. There is
then a lookup table for the color mapping. The noise in the gif seems to
stem from cutoffs from adjacent color
bins.}\label{answer-colormapped-indexed-images-are-based-on-the-true-color-images-but-binned-into-a-histogram-of-256-colors-instead-of-2563-possible-color-combinations-in-the-truecolor-image.-there-is-then-a-lookup-table-for-the-color-mapping.-the-noise-in-the-gif-seems-to-stem-from-cutoffs-from-adjacent-color-bins.}}

    2(d)

Like you did in Homework 1, use Matlab ?s native indexing to cut a 64
� 64 sub-image out of the original truecolor image. Select a region
where there is a smooth change in intensity or color.s

    \begin{Verbatim}[commandchars=\\\{\}]
{\color{incolor}In [{\color{incolor}174}]:} \PY{n}{eng}\PY{o}{.}\PY{n}{eval}\PY{p}{(}\PY{l+s+s1}{\PYZsq{}}\PY{l+s+s1}{crop = bmp(160:160+64, 300:300+64, :);}\PY{l+s+s1}{\PYZsq{}}\PY{p}{,} \PY{n}{nargout}\PY{o}{=}\PY{l+m+mi}{0}\PY{p}{,} \PY{n}{stdout}\PY{o}{=}\PY{n}{out}\PY{p}{,} \PY{n}{stderr}\PY{o}{=}\PY{n}{err}\PY{p}{)}
\end{Verbatim}


    Explain how you did this. Write down the coordinates you chose. Use the
Matlab function, imresize, to enlarge the small image to 512 � 512.
Use the nearest neighbor mode. Display the new image in a figure window.
The individual pixels should be clearly, distinctly visible as squares.
Use the imwrite(�) function to save the enlarged image as a .bmp file.

\hypertarget{answer-the-point-i-chosed-was-160-300.-i-chosed-it-by-selecting-the-middle-ball-which-is-roughly-in-the-center-of-the-image.}{%
\paragraph{Answer: The point I chosed was (160, 300). I chosed it by
selecting the middle ball, which is roughly in the center of the
image.}\label{answer-the-point-i-chosed-was-160-300.-i-chosed-it-by-selecting-the-middle-ball-which-is-roughly-in-the-center-of-the-image.}}

    \begin{Verbatim}[commandchars=\\\{\}]
{\color{incolor}In [{\color{incolor}175}]:} \PY{n}{eng}\PY{o}{.}\PY{n}{eval}\PY{p}{(}\PY{l+s+s1}{\PYZsq{}}\PY{l+s+s1}{crop\PYZus{}resized\PYZus{}bmp = imresize(crop, [512 512]);}\PY{l+s+s1}{\PYZsq{}}\PY{p}{,} \PY{n}{nargout}\PY{o}{=}\PY{l+m+mi}{0}\PY{p}{,} \PY{n}{stdout}\PY{o}{=}\PY{n}{out}\PY{p}{,} \PY{n}{stderr}\PY{o}{=}\PY{n}{err}\PY{p}{)}
          \PY{n}{eng}\PY{o}{.}\PY{n}{eval}\PY{p}{(}\PY{l+s+s1}{\PYZsq{}}\PY{l+s+s1}{figure; image(crop\PYZus{}resized\PYZus{}bmp); truesize;}\PY{l+s+s1}{\PYZsq{}}\PY{p}{,} \PY{n}{nargout}\PY{o}{=}\PY{l+m+mi}{0}\PY{p}{,} \PY{n}{stdout}\PY{o}{=}\PY{n}{out}\PY{p}{,} \PY{n}{stderr}\PY{o}{=}\PY{n}{err}\PY{p}{)}
          \PY{n}{eng}\PY{o}{.}\PY{n}{eval}\PY{p}{(}\PY{l+s+s1}{\PYZsq{}}\PY{l+s+s1}{imwrite(crop\PYZus{}resized\PYZus{}bmp, }\PY{l+s+se}{\PYZbs{}\PYZsq{}}\PY{l+s+s1}{crop\PYZus{}resized\PYZus{}bmp.bmp}\PY{l+s+se}{\PYZbs{}\PYZsq{}}\PY{l+s+s1}{);}\PY{l+s+s1}{\PYZsq{}}\PY{p}{,} \PY{n}{nargout}\PY{o}{=}\PY{l+m+mi}{0}\PY{p}{,} \PY{n}{stdout}\PY{o}{=}\PY{n}{out}\PY{p}{,} \PY{n}{stderr}\PY{o}{=}\PY{n}{err}\PY{p}{)}
          \PY{n}{img} \PY{o}{=} \PY{n}{mpimg}\PY{o}{.}\PY{n}{imread}\PY{p}{(}\PY{l+s+s1}{\PYZsq{}}\PY{l+s+s1}{crop\PYZus{}resized\PYZus{}bmp.bmp}\PY{l+s+s1}{\PYZsq{}}\PY{p}{)}
          \PY{n}{plt}\PY{o}{.}\PY{n}{imshow}\PY{p}{(}\PY{n}{img}\PY{p}{)}
          \PY{n}{plt}\PY{o}{.}\PY{n}{title}\PY{p}{(}\PY{l+s+s1}{\PYZsq{}}\PY{l+s+s1}{2(d) truecolor}\PY{l+s+s1}{\PYZsq{}}\PY{p}{)}
          \PY{n}{plt}\PY{o}{.}\PY{n}{show}\PY{p}{(}\PY{p}{)}
\end{Verbatim}


    \begin{center}
    \adjustimage{max size={0.9\linewidth}{0.9\paperheight}}{output_50_0.png}
    \end{center}
    { \hspace*{\fill} \\}
    
    Do exactly the same thing with cwheel\_mapped.gif using the same image
coordinates, so that you get exactly the same regions out of both
images. Save the enlarged region as a BMP image with the original color
map from cwheel\_mapped.gif to include in your report. If the enlarged
region of the colormapped image is in array J512, then this will save
the image with the colormap as a .bmp file named
cwheel\_mapped\_detail.bmp:

    \begin{Verbatim}[commandchars=\\\{\}]
{\color{incolor}In [{\color{incolor}177}]:} \PY{n}{eng}\PY{o}{.}\PY{n}{eval}\PY{p}{(}\PY{l+s+s1}{\PYZsq{}}\PY{l+s+s1}{crop = gif(160:160+64, 300:300+64, :);}\PY{l+s+s1}{\PYZsq{}}\PY{p}{,} \PY{n}{nargout}\PY{o}{=}\PY{l+m+mi}{0}\PY{p}{,} \PY{n}{stdout}\PY{o}{=}\PY{n}{out}\PY{p}{,} \PY{n}{stderr}\PY{o}{=}\PY{n}{err}\PY{p}{)}
          \PY{n}{eng}\PY{o}{.}\PY{n}{eval}\PY{p}{(}\PY{l+s+s1}{\PYZsq{}}\PY{l+s+s1}{crop\PYZus{}resized\PYZus{}gif = imresize(crop, [512 512]);}\PY{l+s+s1}{\PYZsq{}}\PY{p}{,} \PY{n}{nargout}\PY{o}{=}\PY{l+m+mi}{0}\PY{p}{,} \PY{n}{stdout}\PY{o}{=}\PY{n}{out}\PY{p}{,} \PY{n}{stderr}\PY{o}{=}\PY{n}{err}\PY{p}{)}
          \PY{n}{eng}\PY{o}{.}\PY{n}{eval}\PY{p}{(}\PY{l+s+s1}{\PYZsq{}}\PY{l+s+s1}{figure; image(crop\PYZus{}resized\PYZus{}gif); truesize;}\PY{l+s+s1}{\PYZsq{}}\PY{p}{,} \PY{n}{nargout}\PY{o}{=}\PY{l+m+mi}{0}\PY{p}{,} \PY{n}{stdout}\PY{o}{=}\PY{n}{out}\PY{p}{,} \PY{n}{stderr}\PY{o}{=}\PY{n}{err}\PY{p}{)}
          \PY{n}{eng}\PY{o}{.}\PY{n}{eval}\PY{p}{(}\PY{l+s+s1}{\PYZsq{}}\PY{l+s+s1}{colormap(cmap\PYZus{}gif);}\PY{l+s+s1}{\PYZsq{}}\PY{p}{,} \PY{n}{nargout}\PY{o}{=}\PY{l+m+mi}{0}\PY{p}{,} \PY{n}{stdout}\PY{o}{=}\PY{n}{out}\PY{p}{,} \PY{n}{stderr}\PY{o}{=}\PY{n}{err}\PY{p}{)}
          \PY{n}{eng}\PY{o}{.}\PY{n}{eval}\PY{p}{(}\PY{l+s+s1}{\PYZsq{}}\PY{l+s+s1}{imwrite(crop\PYZus{}resized\PYZus{}gif, cmap\PYZus{}gif, }\PY{l+s+se}{\PYZbs{}\PYZsq{}}\PY{l+s+s1}{crop\PYZus{}resized\PYZus{}gif.bmp}\PY{l+s+se}{\PYZbs{}\PYZsq{}}\PY{l+s+s1}{)}\PY{l+s+s1}{\PYZsq{}}\PY{p}{,} \PY{n}{nargout}\PY{o}{=}\PY{l+m+mi}{0}\PY{p}{,} \PY{n}{stdout}\PY{o}{=}\PY{n}{out}\PY{p}{,} \PY{n}{stderr}\PY{o}{=}\PY{n}{err}\PY{p}{)}
          \PY{n}{img} \PY{o}{=} \PY{n}{mpimg}\PY{o}{.}\PY{n}{imread}\PY{p}{(}\PY{l+s+s1}{\PYZsq{}}\PY{l+s+s1}{crop\PYZus{}resized\PYZus{}gif.bmp}\PY{l+s+s1}{\PYZsq{}}\PY{p}{)}
          \PY{n}{plt}\PY{o}{.}\PY{n}{imshow}\PY{p}{(}\PY{n}{img}\PY{p}{)}
          \PY{n}{plt}\PY{o}{.}\PY{n}{title}\PY{p}{(}\PY{l+s+s1}{\PYZsq{}}\PY{l+s+s1}{2(d) colormapped}\PY{l+s+s1}{\PYZsq{}}\PY{p}{)}
          \PY{n}{plt}\PY{o}{.}\PY{n}{show}\PY{p}{(}\PY{p}{)}
\end{Verbatim}


    \begin{center}
    \adjustimage{max size={0.9\linewidth}{0.9\paperheight}}{output_52_0.png}
    \end{center}
    { \hspace*{\fill} \\}
    
    2(e)

It may appear that the color mapped image is inferior to the truecolor
version. That is usually so for images that are color photographs. But
if the image is a graphic with no more than the number of colors that
can be displayed with an indexed image then the indexed (color mapped)
version may be better, depending on the file format in which it is
stored.

The following image illustrates another situation in which a color
mapping can be useful. Download plume\_gs.bmp from Blackboard, read and
display it and its color map in Matlab.

This image of a smoke plume is such that the intensity (grayscale) of
the image at a pixel is directly proportional to the density of the
smoke at the corresponding location in space. Display the image in two
other figure windows so that you have three copies of it.

Apply the the color map you loaded with the image to one of figures.
Apply the jet(256) color map to the second figure.

    \begin{Verbatim}[commandchars=\\\{\}]
{\color{incolor}In [{\color{incolor}189}]:} \PY{n}{eng}\PY{o}{.}\PY{n}{eval}\PY{p}{(}\PY{l+s+s1}{\PYZsq{}}\PY{l+s+s1}{[plume, cmap\PYZus{}plume] = imread(}\PY{l+s+se}{\PYZbs{}\PYZsq{}}\PY{l+s+s1}{plume\PYZus{}gs.bmp}\PY{l+s+se}{\PYZbs{}\PYZsq{}}\PY{l+s+s1}{);}\PY{l+s+s1}{\PYZsq{}}\PY{p}{,} \PY{n}{nargout}\PY{o}{=}\PY{l+m+mi}{0}\PY{p}{,} \PY{n}{stdout}\PY{o}{=}\PY{n}{out}\PY{p}{,} \PY{n}{stderr}\PY{o}{=}\PY{n}{err}\PY{p}{)}
          \PY{n}{eng}\PY{o}{.}\PY{n}{eval}\PY{p}{(}\PY{l+s+s1}{\PYZsq{}}\PY{l+s+s1}{figure; image(plume); truesize;}\PY{l+s+s1}{\PYZsq{}}\PY{p}{,} \PY{n}{nargout}\PY{o}{=}\PY{l+m+mi}{0}\PY{p}{,} \PY{n}{stdout}\PY{o}{=}\PY{n}{out}\PY{p}{,} \PY{n}{stderr}\PY{o}{=}\PY{n}{err}\PY{p}{)}
          \PY{n}{eng}\PY{o}{.}\PY{n}{eval}\PY{p}{(}\PY{l+s+s1}{\PYZsq{}}\PY{l+s+s1}{set(gca,}\PY{l+s+se}{\PYZbs{}\PYZsq{}}\PY{l+s+s1}{position}\PY{l+s+se}{\PYZbs{}\PYZsq{}}\PY{l+s+s1}{,[0 0 1 1],}\PY{l+s+se}{\PYZbs{}\PYZsq{}}\PY{l+s+s1}{units}\PY{l+s+se}{\PYZbs{}\PYZsq{}}\PY{l+s+s1}{,}\PY{l+s+se}{\PYZbs{}\PYZsq{}}\PY{l+s+s1}{normalized}\PY{l+s+se}{\PYZbs{}\PYZsq{}}\PY{l+s+s1}{, }\PY{l+s+se}{\PYZbs{}\PYZsq{}}\PY{l+s+s1}{Visible}\PY{l+s+se}{\PYZbs{}\PYZsq{}}\PY{l+s+s1}{, }\PY{l+s+se}{\PYZbs{}\PYZsq{}}\PY{l+s+s1}{off}\PY{l+s+se}{\PYZbs{}\PYZsq{}}\PY{l+s+s1}{);}\PY{l+s+s1}{\PYZsq{}}\PY{p}{,} \PY{n}{nargout}\PY{o}{=}\PY{l+m+mi}{0}\PY{p}{,} \PY{n}{stdout}\PY{o}{=}\PY{n}{out}\PY{p}{,} \PY{n}{stderr}\PY{o}{=}\PY{n}{err}\PY{p}{)}
          \PY{n}{eng}\PY{o}{.}\PY{n}{eval}\PY{p}{(}\PY{l+s+s1}{\PYZsq{}}\PY{l+s+s1}{figure; image(plume); truesize;}\PY{l+s+s1}{\PYZsq{}}\PY{p}{,} \PY{n}{nargout}\PY{o}{=}\PY{l+m+mi}{0}\PY{p}{,} \PY{n}{stdout}\PY{o}{=}\PY{n}{out}\PY{p}{,} \PY{n}{stderr}\PY{o}{=}\PY{n}{err}\PY{p}{)}
          \PY{n}{eng}\PY{o}{.}\PY{n}{eval}\PY{p}{(}\PY{l+s+s1}{\PYZsq{}}\PY{l+s+s1}{colormap(cmap\PYZus{}plume);}\PY{l+s+s1}{\PYZsq{}}\PY{p}{,} \PY{n}{nargout}\PY{o}{=}\PY{l+m+mi}{0}\PY{p}{,} \PY{n}{stdout}\PY{o}{=}\PY{n}{out}\PY{p}{,} \PY{n}{stderr}\PY{o}{=}\PY{n}{err}\PY{p}{)}
          \PY{n}{eng}\PY{o}{.}\PY{n}{eval}\PY{p}{(}\PY{l+s+s1}{\PYZsq{}}\PY{l+s+s1}{set(gca,}\PY{l+s+se}{\PYZbs{}\PYZsq{}}\PY{l+s+s1}{position}\PY{l+s+se}{\PYZbs{}\PYZsq{}}\PY{l+s+s1}{,[0 0 1 1],}\PY{l+s+se}{\PYZbs{}\PYZsq{}}\PY{l+s+s1}{units}\PY{l+s+se}{\PYZbs{}\PYZsq{}}\PY{l+s+s1}{,}\PY{l+s+se}{\PYZbs{}\PYZsq{}}\PY{l+s+s1}{normalized}\PY{l+s+se}{\PYZbs{}\PYZsq{}}\PY{l+s+s1}{, }\PY{l+s+se}{\PYZbs{}\PYZsq{}}\PY{l+s+s1}{Visible}\PY{l+s+se}{\PYZbs{}\PYZsq{}}\PY{l+s+s1}{, }\PY{l+s+se}{\PYZbs{}\PYZsq{}}\PY{l+s+s1}{off}\PY{l+s+se}{\PYZbs{}\PYZsq{}}\PY{l+s+s1}{);}\PY{l+s+s1}{\PYZsq{}}\PY{p}{,} \PY{n}{nargout}\PY{o}{=}\PY{l+m+mi}{0}\PY{p}{,} \PY{n}{stdout}\PY{o}{=}\PY{n}{out}\PY{p}{,} \PY{n}{stderr}\PY{o}{=}\PY{n}{err}\PY{p}{)}
          \PY{n}{eng}\PY{o}{.}\PY{n}{eval}\PY{p}{(}\PY{l+s+s1}{\PYZsq{}}\PY{l+s+s1}{figure; image(plume); truesize;}\PY{l+s+s1}{\PYZsq{}}\PY{p}{,} \PY{n}{nargout}\PY{o}{=}\PY{l+m+mi}{0}\PY{p}{,} \PY{n}{stdout}\PY{o}{=}\PY{n}{out}\PY{p}{,} \PY{n}{stderr}\PY{o}{=}\PY{n}{err}\PY{p}{)}
          \PY{n}{eng}\PY{o}{.}\PY{n}{eval}\PY{p}{(}\PY{l+s+s1}{\PYZsq{}}\PY{l+s+s1}{colormap(jet(256));}\PY{l+s+s1}{\PYZsq{}}\PY{p}{,} \PY{n}{nargout}\PY{o}{=}\PY{l+m+mi}{0}\PY{p}{,} \PY{n}{stdout}\PY{o}{=}\PY{n}{out}\PY{p}{,} \PY{n}{stderr}\PY{o}{=}\PY{n}{err}\PY{p}{)}
          \PY{n}{eng}\PY{o}{.}\PY{n}{eval}\PY{p}{(}\PY{l+s+s1}{\PYZsq{}}\PY{l+s+s1}{set(gca,}\PY{l+s+se}{\PYZbs{}\PYZsq{}}\PY{l+s+s1}{position}\PY{l+s+se}{\PYZbs{}\PYZsq{}}\PY{l+s+s1}{,[0 0 1 1],}\PY{l+s+se}{\PYZbs{}\PYZsq{}}\PY{l+s+s1}{units}\PY{l+s+se}{\PYZbs{}\PYZsq{}}\PY{l+s+s1}{,}\PY{l+s+se}{\PYZbs{}\PYZsq{}}\PY{l+s+s1}{normalized}\PY{l+s+se}{\PYZbs{}\PYZsq{}}\PY{l+s+s1}{, }\PY{l+s+se}{\PYZbs{}\PYZsq{}}\PY{l+s+s1}{Visible}\PY{l+s+se}{\PYZbs{}\PYZsq{}}\PY{l+s+s1}{, }\PY{l+s+se}{\PYZbs{}\PYZsq{}}\PY{l+s+s1}{off}\PY{l+s+se}{\PYZbs{}\PYZsq{}}\PY{l+s+s1}{);}\PY{l+s+s1}{\PYZsq{}}\PY{p}{,} \PY{n}{nargout}\PY{o}{=}\PY{l+m+mi}{0}\PY{p}{,} \PY{n}{stdout}\PY{o}{=}\PY{n}{out}\PY{p}{,} \PY{n}{stderr}\PY{o}{=}\PY{n}{err}\PY{p}{)}
          
          \PY{n}{fig}\PY{p}{,} \PY{n}{axes} \PY{o}{=} \PY{n}{plt}\PY{o}{.}\PY{n}{subplots}\PY{p}{(}\PY{l+m+mi}{1}\PY{p}{,} \PY{l+m+mi}{3}\PY{p}{,} \PY{n}{figsize}\PY{o}{=}\PY{p}{(}\PY{l+m+mi}{15}\PY{p}{,} \PY{l+m+mi}{8}\PY{p}{)}\PY{p}{)}
          \PY{n}{img} \PY{o}{=} \PY{n}{mpimg}\PY{o}{.}\PY{n}{imread}\PY{p}{(}\PY{l+s+s1}{\PYZsq{}}\PY{l+s+s1}{2e\PYZus{}plume\PYZus{}no\PYZus{}map.bmp}\PY{l+s+s1}{\PYZsq{}}\PY{p}{)}
          \PY{n}{axes}\PY{p}{[}\PY{l+m+mi}{0}\PY{p}{]}\PY{o}{.}\PY{n}{imshow}\PY{p}{(}\PY{n}{img}\PY{p}{)}
          \PY{n}{axes}\PY{p}{[}\PY{l+m+mi}{0}\PY{p}{]}\PY{o}{.}\PY{n}{set\PYZus{}title}\PY{p}{(}\PY{l+s+s1}{\PYZsq{}}\PY{l+s+s1}{2(e) plume without colormap}\PY{l+s+s1}{\PYZsq{}}\PY{p}{)}
          \PY{n}{img} \PY{o}{=} \PY{n}{mpimg}\PY{o}{.}\PY{n}{imread}\PY{p}{(}\PY{l+s+s1}{\PYZsq{}}\PY{l+s+s1}{2e\PYZus{}plume\PYZus{}map.bmp}\PY{l+s+s1}{\PYZsq{}}\PY{p}{)}
          \PY{n}{axes}\PY{p}{[}\PY{l+m+mi}{1}\PY{p}{]}\PY{o}{.}\PY{n}{imshow}\PY{p}{(}\PY{n}{img}\PY{p}{)}
          \PY{n}{axes}\PY{p}{[}\PY{l+m+mi}{1}\PY{p}{]}\PY{o}{.}\PY{n}{set\PYZus{}title}\PY{p}{(}\PY{l+s+s1}{\PYZsq{}}\PY{l+s+s1}{2(e) plume with jet colormap}\PY{l+s+s1}{\PYZsq{}}\PY{p}{)}
          \PY{n}{img} \PY{o}{=} \PY{n}{mpimg}\PY{o}{.}\PY{n}{imread}\PY{p}{(}\PY{l+s+s1}{\PYZsq{}}\PY{l+s+s1}{2e\PYZus{}plume\PYZus{}jet\PYZus{}256.bmp}\PY{l+s+s1}{\PYZsq{}}\PY{p}{)}
          \PY{n}{axes}\PY{p}{[}\PY{l+m+mi}{2}\PY{p}{]}\PY{o}{.}\PY{n}{imshow}\PY{p}{(}\PY{n}{img}\PY{p}{)}
          \PY{n}{axes}\PY{p}{[}\PY{l+m+mi}{2}\PY{p}{]}\PY{o}{.}\PY{n}{set\PYZus{}title}\PY{p}{(}\PY{l+s+s1}{\PYZsq{}}\PY{l+s+s1}{2(e) plume with colormap}\PY{l+s+s1}{\PYZsq{}}\PY{p}{)}
          \PY{n}{plt}\PY{o}{.}\PY{n}{show}\PY{p}{(}\PY{p}{)}
\end{Verbatim}


    \begin{center}
    \adjustimage{max size={0.9\linewidth}{0.9\paperheight}}{output_54_0.png}
    \end{center}
    { \hspace*{\fill} \\}
    
    2(e) Can you see structures in the plume that were not visible before?
If so, describe them.

\hypertarget{answe-yes.-the-grayscale-colormap-shows-the-inner-structures-with-different-intensities-while-the-jet256-colormap-shows-even-more-distinct-structures-with-different-shades-of-blue-green-and-red.}{%
\paragraph{Answe: Yes. The grayscale colormap shows the inner structures
with different intensities, while the jet(256) colormap shows even more
distinct structures with different shades of blue, green, and
red.}\label{answe-yes.-the-grayscale-colormap-shows-the-inner-structures-with-different-intensities-while-the-jet256-colormap-shows-even-more-distinct-structures-with-different-shades-of-blue-green-and-red.}}

Apply a random color map rand(256,3) to the third figure. Repeat
colormap(rand(256,3)) a few times until you get an image with good
contrast.

    \begin{Verbatim}[commandchars=\\\{\}]
{\color{incolor}In [{\color{incolor}197}]:} \PY{n}{eng}\PY{o}{.}\PY{n}{eval}\PY{p}{(}\PY{l+s+s1}{\PYZsq{}}\PY{l+s+s1}{figure; image(plume); truesize;}\PY{l+s+s1}{\PYZsq{}}\PY{p}{,} \PY{n}{nargout}\PY{o}{=}\PY{l+m+mi}{0}\PY{p}{,} \PY{n}{stdout}\PY{o}{=}\PY{n}{out}\PY{p}{,} \PY{n}{stderr}\PY{o}{=}\PY{n}{err}\PY{p}{)}
          \PY{n}{eng}\PY{o}{.}\PY{n}{eval}\PY{p}{(}\PY{l+s+s1}{\PYZsq{}}\PY{l+s+s1}{colormap(rand(256, 3));}\PY{l+s+s1}{\PYZsq{}}\PY{p}{,} \PY{n}{nargout}\PY{o}{=}\PY{l+m+mi}{0}\PY{p}{,} \PY{n}{stdout}\PY{o}{=}\PY{n}{out}\PY{p}{,} \PY{n}{stderr}\PY{o}{=}\PY{n}{err}\PY{p}{)}
          \PY{n}{eng}\PY{o}{.}\PY{n}{eval}\PY{p}{(}\PY{l+s+s1}{\PYZsq{}}\PY{l+s+s1}{set(gca,}\PY{l+s+se}{\PYZbs{}\PYZsq{}}\PY{l+s+s1}{position}\PY{l+s+se}{\PYZbs{}\PYZsq{}}\PY{l+s+s1}{,[0 0 1 1],}\PY{l+s+se}{\PYZbs{}\PYZsq{}}\PY{l+s+s1}{units}\PY{l+s+se}{\PYZbs{}\PYZsq{}}\PY{l+s+s1}{,}\PY{l+s+se}{\PYZbs{}\PYZsq{}}\PY{l+s+s1}{normalized}\PY{l+s+se}{\PYZbs{}\PYZsq{}}\PY{l+s+s1}{, }\PY{l+s+se}{\PYZbs{}\PYZsq{}}\PY{l+s+s1}{Visible}\PY{l+s+se}{\PYZbs{}\PYZsq{}}\PY{l+s+s1}{, }\PY{l+s+se}{\PYZbs{}\PYZsq{}}\PY{l+s+s1}{off}\PY{l+s+se}{\PYZbs{}\PYZsq{}}\PY{l+s+s1}{);}\PY{l+s+s1}{\PYZsq{}}\PY{p}{,} \PY{n}{nargout}\PY{o}{=}\PY{l+m+mi}{0}\PY{p}{,} \PY{n}{stdout}\PY{o}{=}\PY{n}{out}\PY{p}{,} \PY{n}{stderr}\PY{o}{=}\PY{n}{err}\PY{p}{)}
          
          \PY{n}{img} \PY{o}{=} \PY{n}{mpimg}\PY{o}{.}\PY{n}{imread}\PY{p}{(}\PY{l+s+s1}{\PYZsq{}}\PY{l+s+s1}{2e\PYZus{}plume\PYZus{}words.bmp}\PY{l+s+s1}{\PYZsq{}}\PY{p}{)}
          \PY{n}{plt}\PY{o}{.}\PY{n}{imshow}\PY{p}{(}\PY{n}{img}\PY{p}{)}
          \PY{n}{plt}\PY{o}{.}\PY{n}{title}\PY{p}{(}\PY{l+s+s1}{\PYZsq{}}\PY{l+s+s1}{2(e) plume with random colormap}\PY{l+s+s1}{\PYZsq{}}\PY{p}{)}
          \PY{n}{plt}\PY{o}{.}\PY{n}{show}\PY{p}{(}\PY{p}{)}
\end{Verbatim}


    \begin{center}
    \adjustimage{max size={0.9\linewidth}{0.9\paperheight}}{output_56_0.png}
    \end{center}
    { \hspace*{\fill} \\}
    
    2(e)

Again, can you see structures that were not before visible?

\hypertarget{answer-yes.-the-words-all-your-base-are-belong-to-me}{%
\paragraph{Answer: Yes. The words ``all your base are belong to
me''}\label{answer-yes.-the-words-all-your-base-are-belong-to-me}}

    2(e)

Save the smoke plume image with the jet(256) color map and save it with
the random color map for inclusion in your report.

    \begin{enumerate}
\def\labelenumi{\arabic{enumi}.}
\setcounter{enumi}{2}
\tightlist
\item
  Arithmetic operations on images and sub-images.
\end{enumerate}

3(a) Acquire a 24-bit image about the same size as cwheel.bmp that has a
wide range of intensities (dark to light). Use an image of your choice,
but not cwheel.bmp.

    \begin{Verbatim}[commandchars=\\\{\}]
{\color{incolor}In [{\color{incolor}165}]:} \PY{n}{eng}\PY{o}{.}\PY{n}{eval}\PY{p}{(}\PY{l+s+s1}{\PYZsq{}}\PY{l+s+s1}{merzouga = imread(}\PY{l+s+se}{\PYZbs{}\PYZsq{}}\PY{l+s+s1}{ZhanwenChenMerzouga.JPG}\PY{l+s+se}{\PYZbs{}\PYZsq{}}\PY{l+s+s1}{);}\PY{l+s+s1}{\PYZsq{}}\PY{p}{,} \PY{n}{nargout}\PY{o}{=}\PY{l+m+mi}{0}\PY{p}{,} \PY{n}{stdout}\PY{o}{=}\PY{n}{out}\PY{p}{,} \PY{n}{stderr}\PY{o}{=}\PY{n}{err}\PY{p}{)}
          \PY{n}{eng}\PY{o}{.}\PY{n}{eval}\PY{p}{(}\PY{l+s+s1}{\PYZsq{}}\PY{l+s+s1}{figure; image(merzouga); truesize;}\PY{l+s+s1}{\PYZsq{}}\PY{p}{,} \PY{n}{nargout}\PY{o}{=}\PY{l+m+mi}{0}\PY{p}{,} \PY{n}{stdout}\PY{o}{=}\PY{n}{out}\PY{p}{,} \PY{n}{stderr}\PY{o}{=}\PY{n}{err}\PY{p}{)}
          \PY{n}{eng}\PY{o}{.}\PY{n}{eval}\PY{p}{(}\PY{l+s+s1}{\PYZsq{}}\PY{l+s+s1}{set(gca,}\PY{l+s+se}{\PYZbs{}\PYZsq{}}\PY{l+s+s1}{position}\PY{l+s+se}{\PYZbs{}\PYZsq{}}\PY{l+s+s1}{,[0 0 1 1],}\PY{l+s+se}{\PYZbs{}\PYZsq{}}\PY{l+s+s1}{units}\PY{l+s+se}{\PYZbs{}\PYZsq{}}\PY{l+s+s1}{,}\PY{l+s+se}{\PYZbs{}\PYZsq{}}\PY{l+s+s1}{normalized}\PY{l+s+se}{\PYZbs{}\PYZsq{}}\PY{l+s+s1}{, }\PY{l+s+se}{\PYZbs{}\PYZsq{}}\PY{l+s+s1}{Visible}\PY{l+s+se}{\PYZbs{}\PYZsq{}}\PY{l+s+s1}{, }\PY{l+s+se}{\PYZbs{}\PYZsq{}}\PY{l+s+s1}{off}\PY{l+s+se}{\PYZbs{}\PYZsq{}}\PY{l+s+s1}{);}\PY{l+s+s1}{\PYZsq{}}\PY{p}{,} \PY{n}{nargout}\PY{o}{=}\PY{l+m+mi}{0}\PY{p}{,} \PY{n}{stdout}\PY{o}{=}\PY{n}{out}\PY{p}{,} \PY{n}{stderr}\PY{o}{=}\PY{n}{err}\PY{p}{)}
          \PY{n}{img} \PY{o}{=} \PY{n}{mpimg}\PY{o}{.}\PY{n}{imread}\PY{p}{(}\PY{l+s+s1}{\PYZsq{}}\PY{l+s+s1}{3a\PYZus{}merzouga.bmp}\PY{l+s+s1}{\PYZsq{}}\PY{p}{)}
          \PY{n}{plt}\PY{o}{.}\PY{n}{imshow}\PY{p}{(}\PY{n}{img}\PY{p}{)}
          \PY{n}{plt}\PY{o}{.}\PY{n}{title}\PY{p}{(}\PY{l+s+s1}{\PYZsq{}}\PY{l+s+s1}{3(a) Merzouga, Morocco. ©Zhanwen Chen 2016}\PY{l+s+s1}{\PYZsq{}}\PY{p}{)}
          \PY{n}{plt}\PY{o}{.}\PY{n}{show}\PY{p}{(}\PY{p}{)}
\end{Verbatim}


    \begin{center}
    \adjustimage{max size={0.9\linewidth}{0.9\paperheight}}{output_60_0.png}
    \end{center}
    { \hspace*{\fill} \\}
    
    3(b) Assume matrix J contains your image.

3(b)(i)

Display J/2

    \begin{Verbatim}[commandchars=\\\{\}]
{\color{incolor}In [{\color{incolor}73}]:} \PY{n}{eng}\PY{o}{.}\PY{n}{eval}\PY{p}{(}\PY{l+s+s1}{\PYZsq{}}\PY{l+s+s1}{figure; image(merzouga/2); truesize;}\PY{l+s+s1}{\PYZsq{}}\PY{p}{,} \PY{n}{nargout}\PY{o}{=}\PY{l+m+mi}{0}\PY{p}{,} \PY{n}{stdout}\PY{o}{=}\PY{n}{out}\PY{p}{,} \PY{n}{stderr}\PY{o}{=}\PY{n}{err}\PY{p}{)}
         \PY{n}{eng}\PY{o}{.}\PY{n}{eval}\PY{p}{(}\PY{l+s+s1}{\PYZsq{}}\PY{l+s+s1}{set(gca,}\PY{l+s+se}{\PYZbs{}\PYZsq{}}\PY{l+s+s1}{position}\PY{l+s+se}{\PYZbs{}\PYZsq{}}\PY{l+s+s1}{,[0 0 1 1],}\PY{l+s+se}{\PYZbs{}\PYZsq{}}\PY{l+s+s1}{units}\PY{l+s+se}{\PYZbs{}\PYZsq{}}\PY{l+s+s1}{,}\PY{l+s+se}{\PYZbs{}\PYZsq{}}\PY{l+s+s1}{normalized}\PY{l+s+se}{\PYZbs{}\PYZsq{}}\PY{l+s+s1}{, }\PY{l+s+se}{\PYZbs{}\PYZsq{}}\PY{l+s+s1}{Visible}\PY{l+s+se}{\PYZbs{}\PYZsq{}}\PY{l+s+s1}{, }\PY{l+s+se}{\PYZbs{}\PYZsq{}}\PY{l+s+s1}{off}\PY{l+s+se}{\PYZbs{}\PYZsq{}}\PY{l+s+s1}{);}\PY{l+s+s1}{\PYZsq{}}\PY{p}{,} \PY{n}{nargout}\PY{o}{=}\PY{l+m+mi}{0}\PY{p}{,} \PY{n}{stdout}\PY{o}{=}\PY{n}{out}\PY{p}{,} \PY{n}{stderr}\PY{o}{=}\PY{n}{err}\PY{p}{)}
         
         \PY{n}{img} \PY{o}{=} \PY{n}{mpimg}\PY{o}{.}\PY{n}{imread}\PY{p}{(}\PY{l+s+s1}{\PYZsq{}}\PY{l+s+s1}{3b\PYZus{}i\PYZus{}merzouga\PYZus{}half.bmp}\PY{l+s+s1}{\PYZsq{}}\PY{p}{)}
         \PY{n}{plt}\PY{o}{.}\PY{n}{imshow}\PY{p}{(}\PY{n}{img}\PY{p}{)}
         \PY{n}{plt}\PY{o}{.}\PY{n}{title}\PY{p}{(}\PY{l+s+s1}{\PYZsq{}}\PY{l+s+s1}{3(b)(i)}\PY{l+s+s1}{\PYZsq{}}\PY{p}{)}
         \PY{n}{plt}\PY{o}{.}\PY{n}{show}\PY{p}{(}\PY{p}{)}
\end{Verbatim}


    \begin{center}
    \adjustimage{max size={0.9\linewidth}{0.9\paperheight}}{output_62_0.png}
    \end{center}
    { \hspace*{\fill} \\}
    
    3(b)(i)

\hypertarget{answer-the-halved-image-is-much-darker-for-brighter-pixels-and-a-little-darker-for-already-dark-ones.}{%
\paragraph{Answer: The halved image is much darker for brighter pixels
and a little darker for already dark
ones.}\label{answer-the-halved-image-is-much-darker-for-brighter-pixels-and-a-little-darker-for-already-dark-ones.}}

    3(b)(ii)

Display J*2,

    \begin{Verbatim}[commandchars=\\\{\}]
{\color{incolor}In [{\color{incolor}75}]:} \PY{n}{eng}\PY{o}{.}\PY{n}{eval}\PY{p}{(}\PY{l+s+s1}{\PYZsq{}}\PY{l+s+s1}{figure; image(merzouga*2); truesize;}\PY{l+s+s1}{\PYZsq{}}\PY{p}{,} \PY{n}{nargout}\PY{o}{=}\PY{l+m+mi}{0}\PY{p}{,} \PY{n}{stdout}\PY{o}{=}\PY{n}{out}\PY{p}{,} \PY{n}{stderr}\PY{o}{=}\PY{n}{err}\PY{p}{)}
         \PY{n}{eng}\PY{o}{.}\PY{n}{eval}\PY{p}{(}\PY{l+s+s1}{\PYZsq{}}\PY{l+s+s1}{set(gca,}\PY{l+s+se}{\PYZbs{}\PYZsq{}}\PY{l+s+s1}{position}\PY{l+s+se}{\PYZbs{}\PYZsq{}}\PY{l+s+s1}{,[0 0 1 1],}\PY{l+s+se}{\PYZbs{}\PYZsq{}}\PY{l+s+s1}{units}\PY{l+s+se}{\PYZbs{}\PYZsq{}}\PY{l+s+s1}{,}\PY{l+s+se}{\PYZbs{}\PYZsq{}}\PY{l+s+s1}{normalized}\PY{l+s+se}{\PYZbs{}\PYZsq{}}\PY{l+s+s1}{, }\PY{l+s+se}{\PYZbs{}\PYZsq{}}\PY{l+s+s1}{Visible}\PY{l+s+se}{\PYZbs{}\PYZsq{}}\PY{l+s+s1}{, }\PY{l+s+se}{\PYZbs{}\PYZsq{}}\PY{l+s+s1}{off}\PY{l+s+se}{\PYZbs{}\PYZsq{}}\PY{l+s+s1}{);}\PY{l+s+s1}{\PYZsq{}}\PY{p}{,} \PY{n}{nargout}\PY{o}{=}\PY{l+m+mi}{0}\PY{p}{,} \PY{n}{stdout}\PY{o}{=}\PY{n}{out}\PY{p}{,} \PY{n}{stderr}\PY{o}{=}\PY{n}{err}\PY{p}{)}
         
         \PY{n}{img} \PY{o}{=} \PY{n}{mpimg}\PY{o}{.}\PY{n}{imread}\PY{p}{(}\PY{l+s+s1}{\PYZsq{}}\PY{l+s+s1}{3b\PYZus{}ii\PYZus{}merzouga\PYZus{}double.bmp}\PY{l+s+s1}{\PYZsq{}}\PY{p}{)}
         \PY{n}{plt}\PY{o}{.}\PY{n}{imshow}\PY{p}{(}\PY{n}{img}\PY{p}{)}
         \PY{n}{plt}\PY{o}{.}\PY{n}{title}\PY{p}{(}\PY{l+s+s1}{\PYZsq{}}\PY{l+s+s1}{3(b)(ii)}\PY{l+s+s1}{\PYZsq{}}\PY{p}{)}
         \PY{n}{plt}\PY{o}{.}\PY{n}{show}\PY{p}{(}\PY{p}{)}
\end{Verbatim}


    \begin{center}
    \adjustimage{max size={0.9\linewidth}{0.9\paperheight}}{output_65_0.png}
    \end{center}
    { \hspace*{\fill} \\}
    
    3(b)(ii)

\hypertarget{answer-the-doubled-image-is-much-brighter-for-brighter-pixels-and-a-little-brighter-for-already-bright-ones.}{%
\paragraph{Answer: The doubled image is much brighter for brighter
pixels and a little brighter for already bright
ones.}\label{answer-the-doubled-image-is-much-brighter-for-brighter-pixels-and-a-little-brighter-for-already-bright-ones.}}

    3(b)(iii)

Display J-128

    \begin{Verbatim}[commandchars=\\\{\}]
{\color{incolor}In [{\color{incolor}77}]:} \PY{n}{eng}\PY{o}{.}\PY{n}{eval}\PY{p}{(}\PY{l+s+s1}{\PYZsq{}}\PY{l+s+s1}{figure; image(merzouga\PYZhy{}128); truesize;}\PY{l+s+s1}{\PYZsq{}}\PY{p}{,} \PY{n}{nargout}\PY{o}{=}\PY{l+m+mi}{0}\PY{p}{,} \PY{n}{stdout}\PY{o}{=}\PY{n}{out}\PY{p}{,} \PY{n}{stderr}\PY{o}{=}\PY{n}{err}\PY{p}{)}
         \PY{n}{eng}\PY{o}{.}\PY{n}{eval}\PY{p}{(}\PY{l+s+s1}{\PYZsq{}}\PY{l+s+s1}{set(gca,}\PY{l+s+se}{\PYZbs{}\PYZsq{}}\PY{l+s+s1}{position}\PY{l+s+se}{\PYZbs{}\PYZsq{}}\PY{l+s+s1}{,[0 0 1 1],}\PY{l+s+se}{\PYZbs{}\PYZsq{}}\PY{l+s+s1}{units}\PY{l+s+se}{\PYZbs{}\PYZsq{}}\PY{l+s+s1}{,}\PY{l+s+se}{\PYZbs{}\PYZsq{}}\PY{l+s+s1}{normalized}\PY{l+s+se}{\PYZbs{}\PYZsq{}}\PY{l+s+s1}{, }\PY{l+s+se}{\PYZbs{}\PYZsq{}}\PY{l+s+s1}{Visible}\PY{l+s+se}{\PYZbs{}\PYZsq{}}\PY{l+s+s1}{, }\PY{l+s+se}{\PYZbs{}\PYZsq{}}\PY{l+s+s1}{off}\PY{l+s+se}{\PYZbs{}\PYZsq{}}\PY{l+s+s1}{);}\PY{l+s+s1}{\PYZsq{}}\PY{p}{,} \PY{n}{nargout}\PY{o}{=}\PY{l+m+mi}{0}\PY{p}{,} \PY{n}{stdout}\PY{o}{=}\PY{n}{out}\PY{p}{,} \PY{n}{stderr}\PY{o}{=}\PY{n}{err}\PY{p}{)}
         
         \PY{n}{img} \PY{o}{=} \PY{n}{mpimg}\PY{o}{.}\PY{n}{imread}\PY{p}{(}\PY{l+s+s1}{\PYZsq{}}\PY{l+s+s1}{3b\PYZus{}iii\PYZus{}merzouga\PYZus{}minus.bmp}\PY{l+s+s1}{\PYZsq{}}\PY{p}{)}
         \PY{n}{plt}\PY{o}{.}\PY{n}{imshow}\PY{p}{(}\PY{n}{img}\PY{p}{)}
         \PY{n}{plt}\PY{o}{.}\PY{n}{title}\PY{p}{(}\PY{l+s+s1}{\PYZsq{}}\PY{l+s+s1}{3(b)(iii)}\PY{l+s+s1}{\PYZsq{}}\PY{p}{)}
         \PY{n}{plt}\PY{o}{.}\PY{n}{show}\PY{p}{(}\PY{p}{)}
\end{Verbatim}


    \begin{center}
    \adjustimage{max size={0.9\linewidth}{0.9\paperheight}}{output_68_0.png}
    \end{center}
    { \hspace*{\fill} \\}
    
    3(b)(iii)

\hypertarget{answer-the-minus-image-is-evenly-darker-for-all-pixels.}{%
\paragraph{Answer: The minus image is evenly darker for all
pixels.}\label{answer-the-minus-image-is-evenly-darker-for-all-pixels.}}

    3(b)(iv)

Display J+128

    \begin{Verbatim}[commandchars=\\\{\}]
{\color{incolor}In [{\color{incolor}79}]:} \PY{n}{eng}\PY{o}{.}\PY{n}{eval}\PY{p}{(}\PY{l+s+s1}{\PYZsq{}}\PY{l+s+s1}{figure; image(merzouga+128); truesize;}\PY{l+s+s1}{\PYZsq{}}\PY{p}{,} \PY{n}{nargout}\PY{o}{=}\PY{l+m+mi}{0}\PY{p}{,} \PY{n}{stdout}\PY{o}{=}\PY{n}{out}\PY{p}{,} \PY{n}{stderr}\PY{o}{=}\PY{n}{err}\PY{p}{)}
         \PY{n}{eng}\PY{o}{.}\PY{n}{eval}\PY{p}{(}\PY{l+s+s1}{\PYZsq{}}\PY{l+s+s1}{set(gca,}\PY{l+s+se}{\PYZbs{}\PYZsq{}}\PY{l+s+s1}{position}\PY{l+s+se}{\PYZbs{}\PYZsq{}}\PY{l+s+s1}{,[0 0 1 1],}\PY{l+s+se}{\PYZbs{}\PYZsq{}}\PY{l+s+s1}{units}\PY{l+s+se}{\PYZbs{}\PYZsq{}}\PY{l+s+s1}{,}\PY{l+s+se}{\PYZbs{}\PYZsq{}}\PY{l+s+s1}{normalized}\PY{l+s+se}{\PYZbs{}\PYZsq{}}\PY{l+s+s1}{, }\PY{l+s+se}{\PYZbs{}\PYZsq{}}\PY{l+s+s1}{Visible}\PY{l+s+se}{\PYZbs{}\PYZsq{}}\PY{l+s+s1}{, }\PY{l+s+se}{\PYZbs{}\PYZsq{}}\PY{l+s+s1}{off}\PY{l+s+se}{\PYZbs{}\PYZsq{}}\PY{l+s+s1}{);}\PY{l+s+s1}{\PYZsq{}}\PY{p}{,} \PY{n}{nargout}\PY{o}{=}\PY{l+m+mi}{0}\PY{p}{,} \PY{n}{stdout}\PY{o}{=}\PY{n}{out}\PY{p}{,} \PY{n}{stderr}\PY{o}{=}\PY{n}{err}\PY{p}{)}
         
         \PY{n}{img} \PY{o}{=} \PY{n}{mpimg}\PY{o}{.}\PY{n}{imread}\PY{p}{(}\PY{l+s+s1}{\PYZsq{}}\PY{l+s+s1}{3b\PYZus{}iv\PYZus{}merzouga\PYZus{}plus.bmp}\PY{l+s+s1}{\PYZsq{}}\PY{p}{)}
         \PY{n}{plt}\PY{o}{.}\PY{n}{imshow}\PY{p}{(}\PY{n}{img}\PY{p}{)}
         \PY{n}{plt}\PY{o}{.}\PY{n}{title}\PY{p}{(}\PY{l+s+s1}{\PYZsq{}}\PY{l+s+s1}{3(b)(iv)}\PY{l+s+s1}{\PYZsq{}}\PY{p}{)}
         \PY{n}{plt}\PY{o}{.}\PY{n}{show}\PY{p}{(}\PY{p}{)}
\end{Verbatim}


    \begin{center}
    \adjustimage{max size={0.9\linewidth}{0.9\paperheight}}{output_71_0.png}
    \end{center}
    { \hspace*{\fill} \\}
    
    3(b)(iii)

\hypertarget{answer-the-minus-image-is-evenly-brighter-for-all-pixels.}{%
\paragraph{Answer: The minus image is evenly brighter for all
pixels.}\label{answer-the-minus-image-is-evenly-brighter-for-all-pixels.}}

    3(b)

What happens when the image is of class uint8, and the result of
arithmetic on a pixel would yield a negative value? What happens when
the result is greater than 255?

\hypertarget{answer-because-of-the-uint8-class-negative-values-are-truncated-to-0-and-all-values-greater-than-255-are-truncated-to-255.}{%
\paragraph{Answer: Because of the uint8 class, negative values are
truncated to 0 and all values greater than 255 are truncated to
255.}\label{answer-because-of-the-uint8-class-negative-values-are-truncated-to-0-and-all-values-greater-than-255-are-truncated-to-255.}}

    3(c)

If your image has different dimensions from cwheel.bmp, cut out pieces
of one or both images (using native Matlab indexing) so that sthey are
both have exaactly the same dimensions.

Now read cwheel.bmp into variable S in Matlab.

    \begin{Verbatim}[commandchars=\\\{\}]
{\color{incolor}In [{\color{incolor}81}]:} \PY{n}{eng}\PY{o}{.}\PY{n}{eval}\PY{p}{(}\PY{l+s+s1}{\PYZsq{}}\PY{l+s+s1}{S = imread(}\PY{l+s+se}{\PYZbs{}\PYZsq{}}\PY{l+s+s1}{cwheel.bmp}\PY{l+s+se}{\PYZbs{}\PYZsq{}}\PY{l+s+s1}{);}\PY{l+s+s1}{\PYZsq{}}\PY{p}{,} \PY{n}{nargout}\PY{o}{=}\PY{l+m+mi}{0}\PY{p}{,} \PY{n}{stdout}\PY{o}{=}\PY{n}{out}\PY{p}{,} \PY{n}{stderr}\PY{o}{=}\PY{n}{err}\PY{p}{)}
\end{Verbatim}


    3(c)

Form and display the sum, J+S, and the product,J.* S.

    \begin{Verbatim}[commandchars=\\\{\}]
{\color{incolor}In [{\color{incolor}97}]:} \PY{n}{eng}\PY{o}{.}\PY{n}{eval}\PY{p}{(}\PY{l+s+s1}{\PYZsq{}}\PY{l+s+s1}{summed = merzouga+S;}\PY{l+s+s1}{\PYZsq{}}\PY{p}{,} \PY{n}{nargout}\PY{o}{=}\PY{l+m+mi}{0}\PY{p}{,} \PY{n}{stdout}\PY{o}{=}\PY{n}{out}\PY{p}{,} \PY{n}{stderr}\PY{o}{=}\PY{n}{err}\PY{p}{)}
         \PY{n}{eng}\PY{o}{.}\PY{n}{eval}\PY{p}{(}\PY{l+s+s1}{\PYZsq{}}\PY{l+s+s1}{product = merzouga.*S;}\PY{l+s+s1}{\PYZsq{}}\PY{p}{,} \PY{n}{nargout}\PY{o}{=}\PY{l+m+mi}{0}\PY{p}{,} \PY{n}{stdout}\PY{o}{=}\PY{n}{out}\PY{p}{,} \PY{n}{stderr}\PY{o}{=}\PY{n}{err}\PY{p}{)}
         \PY{n}{eng}\PY{o}{.}\PY{n}{eval}\PY{p}{(}\PY{l+s+s1}{\PYZsq{}}\PY{l+s+s1}{figure; image(summed); truesize;}\PY{l+s+s1}{\PYZsq{}}\PY{p}{,} \PY{n}{nargout}\PY{o}{=}\PY{l+m+mi}{0}\PY{p}{,} \PY{n}{stdout}\PY{o}{=}\PY{n}{out}\PY{p}{,} \PY{n}{stderr}\PY{o}{=}\PY{n}{err}\PY{p}{)}
         \PY{n}{eng}\PY{o}{.}\PY{n}{eval}\PY{p}{(}\PY{l+s+s1}{\PYZsq{}}\PY{l+s+s1}{set(gca,}\PY{l+s+se}{\PYZbs{}\PYZsq{}}\PY{l+s+s1}{position}\PY{l+s+se}{\PYZbs{}\PYZsq{}}\PY{l+s+s1}{,[0 0 1 1],}\PY{l+s+se}{\PYZbs{}\PYZsq{}}\PY{l+s+s1}{units}\PY{l+s+se}{\PYZbs{}\PYZsq{}}\PY{l+s+s1}{,}\PY{l+s+se}{\PYZbs{}\PYZsq{}}\PY{l+s+s1}{normalized}\PY{l+s+se}{\PYZbs{}\PYZsq{}}\PY{l+s+s1}{, }\PY{l+s+se}{\PYZbs{}\PYZsq{}}\PY{l+s+s1}{Visible}\PY{l+s+se}{\PYZbs{}\PYZsq{}}\PY{l+s+s1}{, }\PY{l+s+se}{\PYZbs{}\PYZsq{}}\PY{l+s+s1}{off}\PY{l+s+se}{\PYZbs{}\PYZsq{}}\PY{l+s+s1}{);}\PY{l+s+s1}{\PYZsq{}}\PY{p}{,} \PY{n}{nargout}\PY{o}{=}\PY{l+m+mi}{0}\PY{p}{,} \PY{n}{stdout}\PY{o}{=}\PY{n}{out}\PY{p}{,} \PY{n}{stderr}\PY{o}{=}\PY{n}{err}\PY{p}{)}
         \PY{n}{eng}\PY{o}{.}\PY{n}{eval}\PY{p}{(}\PY{l+s+s1}{\PYZsq{}}\PY{l+s+s1}{figure; image(product); truesize;}\PY{l+s+s1}{\PYZsq{}}\PY{p}{,} \PY{n}{nargout}\PY{o}{=}\PY{l+m+mi}{0}\PY{p}{,} \PY{n}{stdout}\PY{o}{=}\PY{n}{out}\PY{p}{,} \PY{n}{stderr}\PY{o}{=}\PY{n}{err}\PY{p}{)}
         \PY{n}{eng}\PY{o}{.}\PY{n}{eval}\PY{p}{(}\PY{l+s+s1}{\PYZsq{}}\PY{l+s+s1}{set(gca,}\PY{l+s+se}{\PYZbs{}\PYZsq{}}\PY{l+s+s1}{position}\PY{l+s+se}{\PYZbs{}\PYZsq{}}\PY{l+s+s1}{,[0 0 1 1],}\PY{l+s+se}{\PYZbs{}\PYZsq{}}\PY{l+s+s1}{units}\PY{l+s+se}{\PYZbs{}\PYZsq{}}\PY{l+s+s1}{,}\PY{l+s+se}{\PYZbs{}\PYZsq{}}\PY{l+s+s1}{normalized}\PY{l+s+se}{\PYZbs{}\PYZsq{}}\PY{l+s+s1}{, }\PY{l+s+se}{\PYZbs{}\PYZsq{}}\PY{l+s+s1}{Visible}\PY{l+s+se}{\PYZbs{}\PYZsq{}}\PY{l+s+s1}{, }\PY{l+s+se}{\PYZbs{}\PYZsq{}}\PY{l+s+s1}{off}\PY{l+s+se}{\PYZbs{}\PYZsq{}}\PY{l+s+s1}{);}\PY{l+s+s1}{\PYZsq{}}\PY{p}{,} \PY{n}{nargout}\PY{o}{=}\PY{l+m+mi}{0}\PY{p}{,} \PY{n}{stdout}\PY{o}{=}\PY{n}{out}\PY{p}{,} \PY{n}{stderr}\PY{o}{=}\PY{n}{err}\PY{p}{)}
         
         \PY{n}{fig}\PY{p}{,} \PY{n}{axes} \PY{o}{=} \PY{n}{plt}\PY{o}{.}\PY{n}{subplots}\PY{p}{(}\PY{l+m+mi}{1}\PY{p}{,} \PY{l+m+mi}{2}\PY{p}{,} \PY{n}{figsize}\PY{o}{=}\PY{p}{(}\PY{l+m+mi}{12}\PY{p}{,} \PY{l+m+mi}{6}\PY{p}{)}\PY{p}{)}
         \PY{n}{fig}\PY{o}{.}\PY{n}{suptitle}\PY{p}{(}\PY{l+s+s1}{\PYZsq{}}\PY{l+s+s1}{3(c)}\PY{l+s+s1}{\PYZsq{}}\PY{p}{)}
         
         \PY{n}{img} \PY{o}{=} \PY{n}{mpimg}\PY{o}{.}\PY{n}{imread}\PY{p}{(}\PY{l+s+s1}{\PYZsq{}}\PY{l+s+s1}{3c\PYZus{}merzouga\PYZus{}plus\PYZus{}s.bmp}\PY{l+s+s1}{\PYZsq{}}\PY{p}{)}
         \PY{n}{axes}\PY{p}{[}\PY{l+m+mi}{0}\PY{p}{]}\PY{o}{.}\PY{n}{imshow}\PY{p}{(}\PY{n}{img}\PY{p}{)}
         \PY{n}{axes}\PY{p}{[}\PY{l+m+mi}{0}\PY{p}{]}\PY{o}{.}\PY{n}{set\PYZus{}title}\PY{p}{(}\PY{l+s+s1}{\PYZsq{}}\PY{l+s+s1}{summed}\PY{l+s+s1}{\PYZsq{}}\PY{p}{)}
         
         \PY{n}{img} \PY{o}{=} \PY{n}{mpimg}\PY{o}{.}\PY{n}{imread}\PY{p}{(}\PY{l+s+s1}{\PYZsq{}}\PY{l+s+s1}{3c\PYZus{}merzouga\PYZus{}times\PYZus{}s.bmp}\PY{l+s+s1}{\PYZsq{}}\PY{p}{)}
         \PY{n}{axes}\PY{p}{[}\PY{l+m+mi}{1}\PY{p}{]}\PY{o}{.}\PY{n}{imshow}\PY{p}{(}\PY{n}{img}\PY{p}{)}
         \PY{n}{axes}\PY{p}{[}\PY{l+m+mi}{1}\PY{p}{]}\PY{o}{.}\PY{n}{set\PYZus{}title}\PY{p}{(}\PY{l+s+s1}{\PYZsq{}}\PY{l+s+s1}{product}\PY{l+s+s1}{\PYZsq{}}\PY{p}{)}
         
         \PY{n}{fig}\PY{o}{.}\PY{n}{tight\PYZus{}layout}\PY{p}{(}\PY{p}{)}
         \PY{n}{fig}\PY{o}{.}\PY{n}{subplots\PYZus{}adjust}\PY{p}{(}\PY{n}{top}\PY{o}{=}\PY{l+m+mf}{0.88}\PY{p}{)}
         
         \PY{n}{plt}\PY{o}{.}\PY{n}{show}\PY{p}{(}\PY{p}{)}
\end{Verbatim}


    \begin{center}
    \adjustimage{max size={0.9\linewidth}{0.9\paperheight}}{output_77_0.png}
    \end{center}
    { \hspace*{\fill} \\}
    
    3(c)

Describe the results.

\hypertarget{answer-the-sum-image-retains-mostly-the-color-profile-of-the-cwheel-image.-however-there-are-white-spots-where-the-user-provided-image-merzouga-is-most-bright.-as-for-the-product-image-it-became-a-unary-image-where-all-pixel-values-seem-to-be-exactly-255-resulting-in-only-four-colors-cyan-magenta-yellow-and-white.}{%
\paragraph{Answer: The sum image retains mostly the color profile of the
cwheel image. However, there are white spots where the user-provided
image (`merzouga') is most bright. As for the product image, it became a
unary image where all pixel values seem to be exactly 255, resulting in
only four colors: cyan, magenta, yellow, and
white.}\label{answer-the-sum-image-retains-mostly-the-color-profile-of-the-cwheel-image.-however-there-are-white-spots-where-the-user-provided-image-merzouga-is-most-bright.-as-for-the-product-image-it-became-a-unary-image-where-all-pixel-values-seem-to-be-exactly-255-resulting-in-only-four-colors-cyan-magenta-yellow-and-white.}}

    3(c)

What is the maximum possible value of a sum of two images?

    \begin{Verbatim}[commandchars=\\\{\}]
{\color{incolor}In [{\color{incolor}102}]:} \PY{n}{clean\PYZus{}buffers}\PY{p}{(}\PY{n}{out}\PY{p}{,} \PY{n}{err}\PY{p}{)}
          \PY{n}{eng}\PY{o}{.}\PY{n}{eval}\PY{p}{(}\PY{l+s+s1}{\PYZsq{}}\PY{l+s+s1}{sum\PYZus{}double = double(merzouga) + double(S); fprintf(}\PY{l+s+se}{\PYZbs{}\PYZsq{}}\PY{l+s+s1}{3(c) max(sum\PYZus{}double(:)) = }\PY{l+s+si}{\PYZpc{}d}\PY{l+s+se}{\PYZbs{}\PYZbs{}}\PY{l+s+s1}{n}\PY{l+s+se}{\PYZbs{}\PYZsq{}}\PY{l+s+s1}{, max(sum\PYZus{}double(:)));}\PY{l+s+s1}{\PYZsq{}}\PY{p}{,} \PY{n}{nargout}\PY{o}{=}\PY{l+m+mi}{0}\PY{p}{,} \PY{n}{stdout}\PY{o}{=}\PY{n}{out}\PY{p}{,} \PY{n}{stderr}\PY{o}{=}\PY{n}{err}\PY{p}{)}
          \PY{n}{sys}\PY{o}{.}\PY{n}{stdout}\PY{o}{.}\PY{n}{write}\PY{p}{(}\PY{n}{out}\PY{o}{.}\PY{n}{getvalue}\PY{p}{(}\PY{p}{)}\PY{p}{)}
\end{Verbatim}


    \begin{Verbatim}[commandchars=\\\{\}]
3(c) max(sum\_double(:)) = 510

    \end{Verbatim}

    How might you adjust the sum so that the maximum fits in an 8-bit image?

\hypertarget{answer-convert-it-to-double-divide-by-its-max-510-and-convert-back-to-uint8.}{%
\paragraph{Answer: convert it to double, divide by its max (510), and
convert back to
uint8.}\label{answer-convert-it-to-double-divide-by-its-max-510-and-convert-back-to-uint8.}}

What is the maximum possible value of a product of two images?

    \begin{Verbatim}[commandchars=\\\{\}]
{\color{incolor}In [{\color{incolor}104}]:} \PY{n}{clean\PYZus{}buffers}\PY{p}{(}\PY{n}{out}\PY{p}{,} \PY{n}{err}\PY{p}{)}
          \PY{n}{eng}\PY{o}{.}\PY{n}{eval}\PY{p}{(}\PY{l+s+s1}{\PYZsq{}}\PY{l+s+s1}{product\PYZus{}double = double(merzouga).* double(S); fprintf(}\PY{l+s+se}{\PYZbs{}\PYZsq{}}\PY{l+s+s1}{3(c) max(product\PYZus{}double(:)) = }\PY{l+s+si}{\PYZpc{}d}\PY{l+s+se}{\PYZbs{}\PYZbs{}}\PY{l+s+s1}{n}\PY{l+s+se}{\PYZbs{}\PYZsq{}}\PY{l+s+s1}{, max(product\PYZus{}double(:)));}\PY{l+s+s1}{\PYZsq{}}\PY{p}{,} \PY{n}{nargout}\PY{o}{=}\PY{l+m+mi}{0}\PY{p}{,} \PY{n}{stdout}\PY{o}{=}\PY{n}{out}\PY{p}{,} \PY{n}{stderr}\PY{o}{=}\PY{n}{err}\PY{p}{)}
          \PY{n}{sys}\PY{o}{.}\PY{n}{stdout}\PY{o}{.}\PY{n}{write}\PY{p}{(}\PY{n}{out}\PY{o}{.}\PY{n}{getvalue}\PY{p}{(}\PY{p}{)}\PY{p}{)}
\end{Verbatim}


    \begin{Verbatim}[commandchars=\\\{\}]
3(c) max(product\_double(:)) = 65025

    \end{Verbatim}

    How might you adjust the product so that the maximum fits in an 8-bit
image?

\hypertarget{answer-convert-it-to-double-divide-by-its-max-65025-and-convert-back-to-uint8.}{%
\paragraph{Answer: convert it to double, divide by its max (65025), and
convert back to
uint8.}\label{answer-convert-it-to-double-divide-by-its-max-65025-and-convert-back-to-uint8.}}

    \begin{enumerate}
\def\labelenumi{\arabic{enumi}.}
\setcounter{enumi}{3}
\tightlist
\item
  Looping in Matlab
\end{enumerate}

If you are not familiar with Matlab functions, then please read the help
pages for ``function,'' in ``MATLAB \textgreater{} Programming Scripts
and Functions \textgreater{} Functions \textgreater{} Function Basics.''

Matlab provides a way to time the execution of its code, the pair of
functions, tic and toc. They are placed in a function before and after
the code to be timed. For example

J = DivIbyConstLoops(I,c) tic \% code to time goes here toc end

\begin{enumerate}
\def\labelenumi{(\alph{enumi})}
\tightlist
\item
  Write a function that divides every pixel in the image by a constant c
  using loops. That is use for statements to make 3 nested loops. For
  example, have the the first loop index the bands. Inside that loop
  index across the columns. And inside that, loop down the rows. Any
  permutation of the order is OK but there need to be three loops. Note
  that in the function before the loops you will need to use the size
  function to get the dimensions of the image.
\end{enumerate}

Place tic on a line before the first for statement. Place toc on a line
after all three loops (the last of 3 end statements). Load an image and
run the function on it. Note the elapsed time.

    \hypertarget{matlab-function}{%
\paragraph{Matlab function:}\label{matlab-function}}

\begin{Shaded}
\begin{Highlighting}[]
\NormalTok{function divved = DivIbyConstLoops(image, constant)}
\NormalTok{    tic}
\NormalTok{    [nrows, ncols, nbands] = size(image);}
\NormalTok{    for r = }\FloatTok{1}\NormalTok{:nrows}
\NormalTok{        for c = }\FloatTok{1}\NormalTok{:ncols}
\NormalTok{            for b = }\FloatTok{1}\NormalTok{:nbands}
\NormalTok{                image(r, c, b) = uint8(double(image(r, c, b))/constant);}
\NormalTok{            end}
\NormalTok{        end}
\NormalTok{    end}
\NormalTok{    toc}
\NormalTok{    divved = image;}
\NormalTok{end}
\end{Highlighting}
\end{Shaded}

    4(b)

Write a function that divides every pixel in the image by a constant c
without using loops. This is particularly simple,

\hypertarget{matlab-function}{%
\paragraph{Matlab function:}\label{matlab-function}}

\begin{Shaded}
\begin{Highlighting}[]
\NormalTok{function J = DivIbyConstNoLoops(I,c)}
\NormalTok{    tic}
\NormalTok{    J=uint8(double(I)/c);}
\NormalTok{    toc}
\NormalTok{end}
\end{Highlighting}
\end{Shaded}

(Note the type conversions.) Load an image and run the function on it.
Note the elapsed time.

    \begin{Verbatim}[commandchars=\\\{\}]
{\color{incolor}In [{\color{incolor}111}]:} \PY{n}{clean\PYZus{}buffers}\PY{p}{(}\PY{n}{out}\PY{p}{,} \PY{n}{err}\PY{p}{)}
          \PY{n}{eng}\PY{o}{.}\PY{n}{eval}\PY{p}{(}\PY{l+s+s1}{\PYZsq{}}\PY{l+s+s1}{merz\PYZus{}divided\PYZus{}loop = DivIbyConstLoops(merzouga, 2); }\PY{l+s+si}{\PYZpc{} lo}\PY{l+s+s1}{op version took 0.043690 seconds.}\PY{l+s+s1}{\PYZsq{}}\PY{p}{,} \PY{n}{nargout}\PY{o}{=}\PY{l+m+mi}{0}\PY{p}{,} \PY{n}{stdout}\PY{o}{=}\PY{n}{out}\PY{p}{,} \PY{n}{stderr}\PY{o}{=}\PY{n}{err}\PY{p}{)}
          \PY{n}{sys}\PY{o}{.}\PY{n}{stdout}\PY{o}{.}\PY{n}{write}\PY{p}{(}\PY{l+s+s1}{\PYZsq{}}\PY{l+s+s1}{4(b) loop version: }\PY{l+s+s1}{\PYZsq{}} \PY{o}{+} \PY{n}{out}\PY{o}{.}\PY{n}{getvalue}\PY{p}{(}\PY{p}{)}\PY{p}{)}
          
          \PY{n}{clean\PYZus{}buffers}\PY{p}{(}\PY{n}{out}\PY{p}{,} \PY{n}{err}\PY{p}{)}
          \PY{n}{eng}\PY{o}{.}\PY{n}{eval}\PY{p}{(}\PY{l+s+s1}{\PYZsq{}}\PY{l+s+s1}{merz\PYZus{}divided\PYZus{}matmul = DivIbyConstNoLoops(merzouga, 2); }\PY{l+s+si}{\PYZpc{} lo}\PY{l+s+s1}{op version took 0.001388 seconds.}\PY{l+s+s1}{\PYZsq{}}\PY{p}{,} \PY{n}{nargout}\PY{o}{=}\PY{l+m+mi}{0}\PY{p}{,} \PY{n}{stdout}\PY{o}{=}\PY{n}{out}\PY{p}{,} \PY{n}{stderr}\PY{o}{=}\PY{n}{err}\PY{p}{)}
          \PY{n}{sys}\PY{o}{.}\PY{n}{stdout}\PY{o}{.}\PY{n}{write}\PY{p}{(}\PY{l+s+s1}{\PYZsq{}}\PY{l+s+s1}{4(b) vectorized version: }\PY{l+s+s1}{\PYZsq{}} \PY{o}{+} \PY{n}{out}\PY{o}{.}\PY{n}{getvalue}\PY{p}{(}\PY{p}{)}\PY{p}{)}
\end{Verbatim}


    \begin{Verbatim}[commandchars=\\\{\}]
4(b) loop version: Elapsed time is 0.767492 seconds.
4(b) vectorized version: Elapsed time is 0.001071 seconds.

    \end{Verbatim}

    4(b)

What do the 2 times tell you about using loops in matlab?

\hypertarget{answer-matlab-is-optimized-for-vectorized-operations-between-a-matrix-and-a-scalar.}{%
\paragraph{Answer: Matlab is optimized for vectorized operations between
a matrix and a
scalar.}\label{answer-matlab-is-optimized-for-vectorized-operations-between-a-matrix-and-a-scalar.}}

    5(a)(i)

Write a function that outputs an image J made from input image I such
that J(r,c,b) is the average of I(r,c,b) and the 8 pixels that surround
it. Use three nested loops, over the bands, the rows, and the columns.

In the function, record the size of I in the three scalars R, C, and B.
Convert I to double. Allocate J as a zero image the same size as I. Loop
over all B bands. So as not to go outside of I when constructing J, loop
the rows from 2 to R-1 and the columns from 2 to C-1. That means the
resulting J will have a 1-pixel wide band of zeros around it. Convert J
to uint8.

Surround outside loop with tic and toc so that when you run the function
it tells you the amount of time it took to execute the loops.

Load a truecolor image of your choice into Matlab . Run your function on
it, note the time and display the image. Include your code, the size of
the image, and the execution time in your report. You need not include
the input image nor the output image.

\hypertarget{matlab-function-loop-version}{%
\paragraph{Matlab function: loop
version:}\label{matlab-function-loop-version}}

\begin{Shaded}
\begin{Highlighting}[]
\NormalTok{function average_image = getAverageImage_loop(original_image)}
    \CommentTok{% Convert to double}
\NormalTok{    original_image_float = double(original_image);}
    \CommentTok{% Get sizes}
\NormalTok{    [num_rows, num_cols, num_bands] = size(original_image_float);}
    \CommentTok{% Allocate a zeros image}
\NormalTok{    average_image = zeros(num_rows, num_cols, num_bands);}

    \CommentTok{% Do the same thing on each band.}
\NormalTok{    tic}
\NormalTok{    for band = }\FloatTok{1}\NormalTok{:num_bands}
        \CommentTok{% for all points in 2D, average 9-blocks. No edge cases.}
\NormalTok{        for row = }\FloatTok{2}\NormalTok{:num_rows-}\FloatTok{1}
\NormalTok{           for col = }\FloatTok{2}\NormalTok{:num_cols-}\FloatTok{1}
\NormalTok{                average_image(row, col, band) = (original_image_float(row-}\FloatTok{1}\NormalTok{, col-}\FloatTok{1}\NormalTok{, band) ...}
\NormalTok{                                                + original_image_float(row-}\FloatTok{1}\NormalTok{, col, band) ...}
\NormalTok{                                                + original_image_float(row-}\FloatTok{1}\NormalTok{, col+}\FloatTok{1}\NormalTok{, band) ...}
\NormalTok{                                                + original_image_float(row, col-}\FloatTok{1}\NormalTok{, band) ...}
\NormalTok{                                                + original_image_float(row, col, band) ...}
\NormalTok{                                                + original_image_float(row, col+}\FloatTok{1}\NormalTok{, band) ...}
\NormalTok{                                                + original_image_float(row+}\FloatTok{1}\NormalTok{, col-}\FloatTok{1}\NormalTok{, band) ...}
\NormalTok{                                                + original_image_float(row+}\FloatTok{1}\NormalTok{, col, band) ...}
\NormalTok{                                                + original_image_float(row+}\FloatTok{1}\NormalTok{, col+}\FloatTok{1}\NormalTok{, band)) / }\FloatTok{9.0}\NormalTok{;}
\NormalTok{           end}
\NormalTok{        end}
\NormalTok{    end}
\NormalTok{    toc}
\NormalTok{    average_image = uint8(average_image);}
\NormalTok{end}
\end{Highlighting}
\end{Shaded}

\hypertarget{matlab-function-vectorized-version}{%
\paragraph{Matlab function: vectorized
version:}\label{matlab-function-vectorized-version}}

\begin{Shaded}
\begin{Highlighting}[]
\NormalTok{function average_image = getAverageImage_vec(I)}
    \CommentTok{% Convert to double}
\NormalTok{    I = double(I);}
    
    \CommentTok{% Get sizes}
\NormalTok{    [R, C, B] = size(I);}
    
\NormalTok{    average_image = zeros(R, C, B);}
\NormalTok{    tic}
\NormalTok{    average_image = (I + ... }\CommentTok{% original}
\NormalTok{                    [zeros(}\FloatTok{1}\NormalTok{,C,B); I(}\FloatTok{1}\NormalTok{:R-}\FloatTok{1}\NormalTok{,:,:)] + ... }\CommentTok{% down}
\NormalTok{                    [I(}\FloatTok{2}\NormalTok{:R,:,:); zeros(}\FloatTok{1}\NormalTok{,C,B)] + ... }\CommentTok{% up}
\NormalTok{                    [zeros(R,}\FloatTok{1}\NormalTok{,B) I(:,}\FloatTok{1}\NormalTok{:C-}\FloatTok{1}\NormalTok{,:)] + ... }\CommentTok{% right}
\NormalTok{                    [I(:,}\FloatTok{2}\NormalTok{:C,:) zeros(R,}\FloatTok{1}\NormalTok{,B)] + ... }\CommentTok{% left}
\NormalTok{                    [zeros(R, }\FloatTok{1}\NormalTok{, B) [zeros(}\FloatTok{1}\NormalTok{,C-}\FloatTok{1}\NormalTok{,B); I(}\FloatTok{1}\NormalTok{:R-}\FloatTok{1}\NormalTok{,}\FloatTok{1}\NormalTok{:C-}\FloatTok{1}\NormalTok{,:)]] + ... }\CommentTok{% down_right}
\NormalTok{                    [[zeros(}\FloatTok{1}\NormalTok{,C-}\FloatTok{1}\NormalTok{,B); I(}\FloatTok{1}\NormalTok{:R-}\FloatTok{1}\NormalTok{,}\FloatTok{2}\NormalTok{:C,:)] zeros(R, }\FloatTok{1}\NormalTok{, B)] + ... }\CommentTok{% down_left}
\NormalTok{                    [zeros(R, }\FloatTok{1}\NormalTok{, B) [I(}\FloatTok{2}\NormalTok{:R,}\FloatTok{1}\NormalTok{:C-}\FloatTok{1}\NormalTok{,:); zeros(}\FloatTok{1}\NormalTok{,C-}\FloatTok{1}\NormalTok{,B)]] + ... }\CommentTok{% up_right}
\NormalTok{                    [[I(}\FloatTok{2}\NormalTok{:R,}\FloatTok{2}\NormalTok{:C,:); zeros(}\FloatTok{1}\NormalTok{,C-}\FloatTok{1}\NormalTok{,B)] zeros(R, }\FloatTok{1}\NormalTok{, B)]) / }\FloatTok{9.0}\NormalTok{; }\CommentTok{% up_left}
\NormalTok{                    average_image([}\FloatTok{1}\NormalTok{, R], :, :) = }\FloatTok{0}\NormalTok{;}
\NormalTok{                    average_image(:, [}\FloatTok{1}\NormalTok{, C], :) = }\FloatTok{0}\NormalTok{;}
\NormalTok{    toc}
\NormalTok{    average_image = uint8(average_image);}
\NormalTok{end}
\end{Highlighting}
\end{Shaded}

    Repeat the two methods serveral times to get a typical execution time.
Were the results what you expected? Explain why the results might be so

\hypertarget{answer-on-average-the-loop-implementation-took-0.021786-seconds-while-the-vectoized-one-took-0.083131-seconds.-these-results-surprised-me-because}{%
\paragraph{Answer: On average, the loop implementation took 0.021786
seconds, while the vectoized one took 0.083131 seconds. These results
surprised me,
because}\label{answer-on-average-the-loop-implementation-took-0.021786-seconds-while-the-vectoized-one-took-0.083131-seconds.-these-results-surprised-me-because}}

    \begin{Verbatim}[commandchars=\\\{\}]
{\color{incolor}In [{\color{incolor}113}]:} \PY{n}{clean\PYZus{}buffers}\PY{p}{(}\PY{n}{out}\PY{p}{,} \PY{n}{err}\PY{p}{)}
          \PY{n}{eng}\PY{o}{.}\PY{n}{eval}\PY{p}{(}\PY{l+s+s2}{\PYZdq{}}\PY{l+s+s2}{loop = getAverageImage\PYZus{}loop(I);}\PY{l+s+s2}{\PYZdq{}}\PY{p}{,} \PY{n}{nargout}\PY{o}{=}\PY{l+m+mi}{0}\PY{p}{,} \PY{n}{stdout}\PY{o}{=}\PY{n}{out}\PY{p}{,} \PY{n}{stderr}\PY{o}{=}\PY{n}{err}\PY{p}{)}
          \PY{n}{sys}\PY{o}{.}\PY{n}{stdout}\PY{o}{.}\PY{n}{write}\PY{p}{(}\PY{l+s+s2}{\PYZdq{}}\PY{l+s+s2}{5(a)(i) loop version: }\PY{l+s+s2}{\PYZdq{}} \PY{o}{+} \PY{n}{out}\PY{o}{.}\PY{n}{getvalue}\PY{p}{(}\PY{p}{)}\PY{p}{)}
          
          \PY{n}{clean\PYZus{}buffers}\PY{p}{(}\PY{n}{out}\PY{p}{,} \PY{n}{err}\PY{p}{)}
          \PY{n}{eng}\PY{o}{.}\PY{n}{eval}\PY{p}{(}\PY{l+s+s2}{\PYZdq{}}\PY{l+s+s2}{vec = getAverageImage\PYZus{}vec(I);}\PY{l+s+s2}{\PYZdq{}}\PY{p}{,} \PY{n}{nargout}\PY{o}{=}\PY{l+m+mi}{0}\PY{p}{,} \PY{n}{stdout}\PY{o}{=}\PY{n}{out}\PY{p}{,} \PY{n}{stderr}\PY{o}{=}\PY{n}{err}\PY{p}{)}
          \PY{n}{sys}\PY{o}{.}\PY{n}{stdout}\PY{o}{.}\PY{n}{write}\PY{p}{(}\PY{l+s+s2}{\PYZdq{}}\PY{l+s+s2}{5(a)(ii) vectorized version: }\PY{l+s+s2}{\PYZdq{}} \PY{o}{+} \PY{n}{out}\PY{o}{.}\PY{n}{getvalue}\PY{p}{(}\PY{p}{)}\PY{p}{)}
\end{Verbatim}


    \begin{Verbatim}[commandchars=\\\{\}]
5(a)(i) loop version: Elapsed time is 0.022869 seconds.
5(a)(ii) vectorized version: Elapsed time is 0.104358 seconds.

    \end{Verbatim}

    5(b)

Write two functions that accept an M x N matrix and output as L2
distance matrix. Assume that the input matrix contains N M-vectos (each
column is a vector). The distance matrix will be N x N. Element (i, j)
contains the L2 (a.k.a. Euclidean) distance between the ith and jth
vectors in the input matrix.

The first program should be written using loops. Note that there are
symmetries in the resultant matrix that can be exploited to reduce
execution time.

\hypertarget{matlab-function-loop-version}{%
\paragraph{Matlab function: loop
version:}\label{matlab-function-loop-version}}

\begin{Shaded}
\begin{Highlighting}[]
\NormalTok{function D2 = getL2Dist_loop(A)}
\NormalTok{    tic}
\NormalTok{    n = size(A, }\FloatTok{2}\NormalTok{);}
\NormalTok{    D2 = zeros(n, n);}
\NormalTok{    for r = }\FloatTok{1}\NormalTok{:n}
\NormalTok{        for c = (r+}\FloatTok{1}\NormalTok{):n}
\NormalTok{            diff = A(:,r)-A(:,c);}
\NormalTok{            D2(r,c) = sqrt(diff'*diff);}
\NormalTok{            if r == }\FloatTok{1000}\NormalTok{ && c == }\FloatTok{11}
\NormalTok{                fprintf(}\StringTok{'(1000, 11): diff = %s, D2(1000, 11) = %f'}\NormalTok{, mat2str(diff), D2(r,c));}
\NormalTok{            end}
\NormalTok{        end}
\NormalTok{    end}
\NormalTok{    D2 = D2 + D2';}
\NormalTok{    toc}
\NormalTok{end}
\end{Highlighting}
\end{Shaded}

The second program should use no loops. That can be accomplished using
bsxfun, but there may be other ways. Consider computing the distance
matrix using the expansion of the L2 distance into an expression in
terms of L2 inner products. I.e., \textbar{}\textbar{}xi
x\textbar{}j\textbar{}\textbar{} = `??'

\hypertarget{matlab-function-vectorized-version}{%
\paragraph{Matlab function: vectorized
version:}\label{matlab-function-vectorized-version}}

\begin{Shaded}
\begin{Highlighting}[]
\NormalTok{function D2 = getL2Dist_vec(A)}
\NormalTok{    tic}
\NormalTok{    D2 = sqrt(bsxfun(@plus,sum(A.^}\FloatTok{2}\NormalTok{)',bsxfun(@plus,sum(A.^}\FloatTok{2}\NormalTok{),-}\FloatTok{2}\NormalTok{*(A'*A))));}
\NormalTok{    toc}
\NormalTok{end}
\end{Highlighting}
\end{Shaded}

Compare the execution times of the two on a random 10 x 1000 matrix
using tic and toc. Compare thos results to that of the Matlab built-in
pdist() from the statistics toolbox. Repeat all three several times to
get a typical time. Were the results what you expected? Explain why the
results might be so.

\hypertarget{answer-looped-version-takes-0.063182-seconds.-vectorized-0.005860-seconds-and-pdist-0.013539-seconds.}{%
\paragraph{Answer: looped version takes 0.063182 seconds. vectorized
0.005860 seconds, and pdist 0.013539
seconds.}\label{answer-looped-version-takes-0.063182-seconds.-vectorized-0.005860-seconds-and-pdist-0.013539-seconds.}}

    \begin{Verbatim}[commandchars=\\\{\}]
{\color{incolor}In [{\color{incolor}117}]:} \PY{n}{eng}\PY{o}{.}\PY{n}{eval}\PY{p}{(}\PY{l+s+s1}{\PYZsq{}}\PY{l+s+s1}{rando = randi([0, 255], 10, 1000);}\PY{l+s+s1}{\PYZsq{}}\PY{p}{,} \PY{n}{nargout}\PY{o}{=}\PY{l+m+mi}{0}\PY{p}{,} \PY{n}{stdout}\PY{o}{=}\PY{n}{out}\PY{p}{,} \PY{n}{stderr}\PY{o}{=}\PY{n}{err}\PY{p}{)}
          
          \PY{n}{clean\PYZus{}buffers}\PY{p}{(}\PY{n}{out}\PY{p}{,} \PY{n}{err}\PY{p}{)}
          \PY{n}{eng}\PY{o}{.}\PY{n}{eval}\PY{p}{(}\PY{l+s+s1}{\PYZsq{}}\PY{l+s+s1}{D\PYZus{}loop = getL2Dist\PYZus{}loop(rando);}\PY{l+s+s1}{\PYZsq{}}\PY{p}{,} \PY{n}{nargout}\PY{o}{=}\PY{l+m+mi}{0}\PY{p}{,} \PY{n}{stdout}\PY{o}{=}\PY{n}{out}\PY{p}{,} \PY{n}{stderr}\PY{o}{=}\PY{n}{err}\PY{p}{)}
          \PY{n}{sys}\PY{o}{.}\PY{n}{stdout}\PY{o}{.}\PY{n}{write}\PY{p}{(}\PY{l+s+s1}{\PYZsq{}}\PY{l+s+s1}{5(b) loop version: }\PY{l+s+s1}{\PYZsq{}} \PY{o}{+} \PY{n}{out}\PY{o}{.}\PY{n}{getvalue}\PY{p}{(}\PY{p}{)}\PY{p}{)}
          
          \PY{n}{clean\PYZus{}buffers}\PY{p}{(}\PY{n}{out}\PY{p}{,} \PY{n}{err}\PY{p}{)}
          \PY{n}{eng}\PY{o}{.}\PY{n}{eval}\PY{p}{(}\PY{l+s+s1}{\PYZsq{}}\PY{l+s+s1}{D\PYZus{}vec = getL2Dist\PYZus{}vec(rando);}\PY{l+s+s1}{\PYZsq{}}\PY{p}{,} \PY{n}{nargout}\PY{o}{=}\PY{l+m+mi}{0}\PY{p}{,} \PY{n}{stdout}\PY{o}{=}\PY{n}{out}\PY{p}{,} \PY{n}{stderr}\PY{o}{=}\PY{n}{err}\PY{p}{)}
          \PY{n}{sys}\PY{o}{.}\PY{n}{stdout}\PY{o}{.}\PY{n}{write}\PY{p}{(}\PY{l+s+s1}{\PYZsq{}}\PY{l+s+s1}{5(b) vectorized version: }\PY{l+s+s1}{\PYZsq{}} \PY{o}{+} \PY{n}{out}\PY{o}{.}\PY{n}{getvalue}\PY{p}{(}\PY{p}{)}\PY{p}{)}
          
          \PY{n}{clean\PYZus{}buffers}\PY{p}{(}\PY{n}{out}\PY{p}{,} \PY{n}{err}\PY{p}{)}
          \PY{n}{eng}\PY{o}{.}\PY{n}{eval}\PY{p}{(}\PY{l+s+s1}{\PYZsq{}}\PY{l+s+s1}{tic; D\PYZus{}pdist = squareform(pdist(rando}\PY{l+s+se}{\PYZbs{}\PYZsq{}}\PY{l+s+s1}{)); toc}\PY{l+s+s1}{\PYZsq{}}\PY{p}{,} \PY{n}{nargout}\PY{o}{=}\PY{l+m+mi}{0}\PY{p}{,} \PY{n}{stdout}\PY{o}{=}\PY{n}{out}\PY{p}{,} \PY{n}{stderr}\PY{o}{=}\PY{n}{err}\PY{p}{)}
          \PY{n}{sys}\PY{o}{.}\PY{n}{stdout}\PY{o}{.}\PY{n}{write}\PY{p}{(}\PY{l+s+s1}{\PYZsq{}}\PY{l+s+s1}{5(b) pdist version: }\PY{l+s+s1}{\PYZsq{}} \PY{o}{+} \PY{n}{out}\PY{o}{.}\PY{n}{getvalue}\PY{p}{(}\PY{p}{)}\PY{p}{)}
\end{Verbatim}


    \begin{Verbatim}[commandchars=\\\{\}]
5(b) loop version: Elapsed time is 0.076565 seconds.
5(b) vectorized version: Elapsed time is 0.010103 seconds.
5(b) pdist version: Elapsed time is 0.019516 seconds.

    \end{Verbatim}


    % Add a bibliography block to the postdoc
    
    
    
    \end{document}
